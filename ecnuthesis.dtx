% \iffalse meta-comment
% !TeX program  = XeLaTeX
% !TeX encoding = UTF-8
%
% Copyright (C) 2021--2023 by Koyamin <koyafumin@gmail.com>
%
% This work may be distributed and/or modified under the
% conditions of the LaTeX Project Public License, either
% version 1.3c of this license or (at your option) any later
% version. The latest version of this license is in:
%
%   http://www.latex-project.org/lppl.txt
%
% and version 1.3 or later is part of all distributions of
% LaTeX version 2005/12/01 or later.
% 
% \fi

% \iffalse
%<class>\NeedsTeXFormat{LaTeX2e}
%<class>\RequirePackage{expl3}
%<class>\GetIdInfo $Id: ecnuthesis.cls 0.2.1 2022-04-20 15:00:00Z Koyamin <koyafumin@gmail.com> Qucheng <i@qcmiao.cn> $
%<class>  {Undergraduate Thesis Template for East China Normal University}
%<class>\ProvidesExplClass{\ExplFileName}
%<class>  {\ExplFileDate}{\ExplFileVersion}{\ExplFileDescription}

%<*driver>
\documentclass{l3doc}
\usepackage{ecnudoc}
\EnableCrossrefs
\CodelineIndex
\RecordChanges
\begin{document}
  \DocInput{ecnuthesis.dtx}
  \PrintIndex
\end{document}
%</driver>
% \fi
%
% \title{\cls{ECNUThesis-Undergraduate}: 华东师范大学本科生毕业论文模板}
% \author{Koyamin}
% \date{\today}
% \maketitle
%
% \begin{abstract}
% \cls{ECNUThesis-Undergraduate} 是一个
% \end{abstract}
%
% \section{介绍}
%
% \cls{ECNUThesis-Undergraduate} (\textbf{E}ast \textbf{C}hina 
% \textbf{N}ormal \textbf{U}niversity \LaTeX{} \textbf{Thesis} 
% Template of \textbf{Undergraduate})是华东师范大学本科生毕业
% 论文的一个 \LaTeX{} 模版。 \cls{ECNUThesis-Undergraduate} 旨在帮助
% 华东师范大学的本科学生使用 \LaTeX{} 完成毕业论文的撰写与排版工作。
%
% \subsection*{开发背景}
%
% 在本模版诞生前,YijunYuan 的 ECNU-Undergraduate-LaTeX 是流传范围最广、
% 使用人数最多的华东师范大学本科生毕业论文 \LaTeX{} 模版。该模版基本满足了
% 本科生毕业论文的格式要求,填补了 \LaTeX{} 版华东师范大学毕业论文模版的空白。
%
% 然而,该模版没有经过系统的设计,没有简明的用户接口,且缺乏后期维护。另一方面,
% 越来越多的华东师范大学学生选择使用 \LaTeX{} 来完成他们的本科生毕业论文,因此
% 有必要开发一个格式规范、接口简明、易于维护的 \LaTeX{} 毕业论文模版。
%
% \subsection*{模版功能}
%
% 本模版使用 \LaTeX3 实现。
% 具体地说,本模版实现了这些功能:
% \begin{itemize}
%   \item 
%   \item 
%   \item 
% \end{itemize}
% 
%
% \section{安装模版}
%
% \subsection{获取 \cls{ECNUThesis-Undergraduate}}
%
% \subsection{模版的组成}
%
%
% \section{使用说明}
%
% \subsection{模版选项}
%
% 模版选项是指在导入文档类的时候指定的选项:
% 
% \begin{latexCode}
%   \documentclass(*\oarg{模版选项}*){ecnuthesis}
% \end{latexCode}
% 
% 一些模板选项为布尔型,它们只能在 \texttt{true} 和 \texttt{false}
% 中取值。对于这些选项,\texttt{\meta{选项}=true} 中的 \texttt{=true}
% 能够被省略。
%
%
% \subsection{参数设置}
%
% \cls{ECNUThesis-Undergraduate} 提供了一系列选项,以帮助用户配置
% 论文格式以及录入相关信息。为了方便用户配置这些选项,
% \cls{ECNUThesis-Undergraduate} 提供了一个简洁的命令。
%
% \begin{function}{\ecnuSetup}
%   \begin{syntax}
%     \cs{ecnuSetup} = \marg{键值对列表}
%   \end{syntax}
%   \indent
%   该命令用于配置相关自定义选项。导入文档类后,下列所有选项均可使用该命令配置。
%   该命令需要写在导言区。
% \end{function}
%
% \cs{ecnuSetup} 的参数是一组由英文逗号(即 \texttt{,})分隔的选项列表,
% 列表中的选项通常是 \texttt{\meta{key}=\meta{value}} 的形式,我们称这种
% 形式为键值,由键值组成的选项列表称为键值列表。键值中的 \texttt{=} 周围的
% 空格不影响设置,但是键值列表之间不可以出现多余的空行。对于同一个选项,后面的
% 设置将会覆盖前一个设置。
%
% 有些选项包含子选项,比如 \texttt{style} 选项和 \texttt{info} 等选项。
% 这时候,以下两种设置方式是等价的:
% \begin{latexCode}
%   \ecnuSetup{
%     style = { fontCJK = fandol, fontMath = times },
%     info  = {
%       title       = {计算机器与智能},
%       titleEN     = {Computing Machinery and Intelligence},
%       author      = {艾伦·麦席森·图灵},
%       department  = {计算机科学与技术学院},
%       major       = {计算机科学与技术}
%     }
%   }
% \end{latexCode}
% 或者
% \begin{latexCode}
%   \ecnuSetup{
%     style/fontCJK   = fandol,
%     style/fontMath  = times,
%     info/title       = {计算机器与智能},
%     info/titleEN     = {Computing Machinery and Intelligence},
%     info/author      = {艾伦·麦席森·图灵},
%     info/department  = {计算机科学与技术学院},
%     info/major       = {计算机科学与技术}
%   }
% \end{latexCode}
%
% 请注意, \texttt{/} 的前后均不可以出现空白字符。
%
%
% \subsubsection{论文基本信息}
%
% \cls{ECNUThesis-Undergraduate} 提供了一系列选项,以帮助用户轻松地
% 填写论文基本信息。
%
% \begin{function}{info}
%   \begin{syntax}
%     info = \marg{键值对列表}
%     info/\meta{key} = \meta{value}
%   \end{syntax}
%   \indent
%   该选项用于录入论文信息,其拥有一些子选项。具体的设置子选项如下所示。
% \end{function}
%
% \begin{function}{info/title, info/titleEN}
%   \begin{syntax}
%     info/title = \marg{中文标题}
%     info/title = \marg{English Title}
%   \end{syntax}
%   \indent
%   论文的中英文标题。可以在标题内部使用换行。
% \end{function}
%% 
% \begin{function}{info/author}
%   \begin{syntax}
%     info/author = \marg{姓名}
%   \end{syntax}
%   \indent
%   作者姓名。
% \end{function}
% 
% \begin{function}{info/studentID}
%   \begin{syntax}
%     info/studentID = \marg{学号}
%   \end{syntax}
%   \indent
%   作者学号。
% \end{function}
%
% \begin{function}{info/department}
%   \begin{syntax}
%     info/department = \marg{名称}
%   \end{syntax}
%   \indent
%   作者所在学院名称。
% \end{function}
%
% \begin{function}{info/major}
%   \begin{syntax}
%     info/major = \marg{名称}
%   \end{syntax}
%   \indent
%   作者所在专业名称。
% \end{function}
%
% \begin{function}{info/supervisor}
%   \begin{syntax}
%     info/supervisor = \marg{姓名}
%   \end{syntax}
%   \indent
%   论文指导教师姓名。
% \end{function}
%
% \begin{function}{info/academicTitle}
%   \begin{syntax}
%     info/academicTitle = \marg{职称}
%   \end{syntax}
%   \indent
%   论文指导教师职称。
% \end{function}
%
% \begin{function}{info/year}
%   \begin{syntax}
%     info/year = \marg{年份}
%   \end{syntax}
%   \indent
%   论文完成年份。若该选项被省略,模版将调用当前年份。请注意,该选项没有花括号。
% \end{function}
%
% \begin{function}{info/month}
%   \begin{syntax}
%     info/month = \marg{月份}
%   \end{syntax}
%   \indent
%   论文完成月份。若该选项被省略,模版将调用当前月份。请注意,该选项没有花括号。
% \end{function}
%
% \begin{function}{info/graduationYear}
%   \begin{syntax}
%     info/graduationYear = \marg{年份}
%   \end{syntax}
%   \indent
%   内封面毕业届别。若该选项被省略,模版将调用 \texttt{year} 字段的值;
%   若 \texttt{year} 选项也被省略,模版将调用当前年份。
%   请注意,该选项没有花括号。
% \end{function}
%
% \begin{function}{info/keywords, info/keywordsEN}
%   \begin{syntax}
%     info/keywords = \marg{关键词}
%     info/keywordsEN = \marg{Keywords}
%   \end{syntax}
%   \indent
%   中英文关键词,请使用英文逗号 “\texttt{,}” 以分隔。
% \end{function}
%
%
% \subsubsection{论文格式设置}
%
% \cls{ECNUThesis-Undergraduate} 提供了一系列选项,以帮助用户轻松地
% 根据自己的需要设置论文格式。
%
% \begin{function}{style}
%   \begin{syntax}
%     style = \marg{键值对列表}
%     style/\meta{key} = \meta{value}
%   \end{syntax}
%   \indent
%   该选项用于设置论文格式相关选项,其拥有一些子选项。具体的设置子选项如下所示。
% \end{function}
%
% \begin{function}{style/numbering}
%   \begin{syntax}
%     style/numbering = alpha{\textbar}\textbf{arabic}{\textbar}chinese
%   \end{syntax}
%   \indent
%   设置论文的章节编号样式。 \texttt{alpha} 使用数字进行编号(即外文要求),
%   \texttt{arabic} 使用数字进行编号(即理科要求),
%   \texttt{chinese} 使用数字进行编号(即文科要求),
%   默认选项为 \texttt{arabic}。
% \end{function}
%
% \begin{function}{style/fontCJK}
%   \begin{syntax}
%     style/fontCJK = \textbf{fandol}{\textbar}mac{\textbar}windows
%   \end{syntax}
%   \indent
%   设置论文的中文字体。默认选项为 \texttt{fandol}。
% \end{function}
%
% \begin{function}{style/fontMath}
%   \begin{syntax}
%     style/fontMath = \textbf{lm}{\textbar}times
%   \end{syntax}
%   \indent
%   设置论文的数学字体。默认选项为 \texttt{lm}。
% \end{function}
%
% \begin{function}{style/footnote}
%   \begin{syntax}
%     style/footnote = circled{\textbar}\textbf{plain}
%   \end{syntax}
%   \indent
%   设置脚注编号样式。\texttt{plain} 表示脚注的编号仅为数字;
%   \texttt{circled} 表示脚注的编号为带圆圈数字 (仅限1-10)。
%   默认选项为 \texttt{plain}。
% \end{function}
%
% \begin{function}{style/bibResource}
%   \begin{syntax}
%     style/bibResource = \marg{文件}
%   \end{syntax}
%   \indent
%   设置参考文献数据源。由于参考文献使用 biber 进行处理,
%   所以必须明确给出数据源文件的 \texttt{.bib} 后缀名。
% \end{function}
%
% \begin{function}{style/logoResource}
%   \begin{syntax}
%     style/logoResource = \marg{文件}
%   \end{syntax}
%   \indent
%   设置封面插图数据源。模版已自带,位于 \texttt{./source/inner-cover(contains\_font).eps}。
% \end{function}
%
% \begin{function}{style/declPageResource}
%   \begin{syntax}
%     style/declPageResource = \marg{文件}
%   \end{syntax}
%   \indent
%   设置待插入的扫描版声明页 PDF 文件。若该值为空则生成模版预定义的声明页;
%   否则将指定路径所对应的 PDF 文件作为声明页插入。
% \end{function}
%
% \subsection{正文编写}
%
% \subsubsection{前置部分}
%
% \DescribeMacro{\frontmatter}
% \cs{frontmatter} 命令用于声明论文前置部分开始。
% 前置部分包含中英文摘要。前置部分的页码采用大写罗马字母,并且与正文分开计数。
%
% \DescribeEnv{abstract}
% \DescribeEnv{abstractEN}
% \env{abstract} 环境用于声明论文前置部分开始。
% 前置部分包含中英文摘要。前置部分的页码采用大写罗马字母,并且与正文分开计数。
%
% \subsubsection{主体部分}
%
% \DescribeMacro{\mainmatter}
% \cs{mainmatter} 命令用于声明论文主体部分开始。
% 主体部分是论文的核心。主体部分可以分章节进行编写,也可以采用多文件编译。
% 主体部分的页码采用阿拉伯数字。
%
% \subsubsection{后置部分}
%
% \DescribeMacro{\backmatter}
% \cs{backmatter} 命令用于声明论文后置部分开始。
% 后置部分包括参考文献、附录以及致谢。后置部分的页码同样使用阿拉伯数字,且与
% 正文部分相接。
%
% \DescribeOption{mole}
%
% \section{宏包依赖情况}
%
% 
%
% \section{致谢}
%
%
%
%
%
%
%
%
%
%
% \section{实现细节}
%
% \cls{ECNUThesis-Undergraduate} 使用 \LaTeX3 语法编写,
% 依赖 \pkg{expl3} 环境,并依赖 \pkg{l3packages} 中的相关宏包。
%
% 根据 \LaTeX3 的语法规则,代码中的空格、换行、回车与制表符将被完全忽略,
% 而下划线 \texttt{\_} 和冒号 \texttt{:} 则可作为一般字母使用。
%
% \subsection{准备}
%
% 
%    \begin{macrocode}
%<@@=ecnu>
%<*class>
%    \end{macrocode}
%
% 导入 \LaTeX3 编程环境。
%
%    \begin{macrocode}
\RequirePackage { xparse, xtemplate, l3keys2e }
%    \end{macrocode}
%
% 目前 \cls{ECNUThesis-Undergraduate} 仅支持 \XeTeX{}。
%    \begin{macrocode}
\msg_new:nnn { ecnuthesis } { engine-unsupportd } 
  { \\
    This~ class~ requires~ XeTeX~ to~ compile. \\
    The~ engine~ "#1"~ is~ not~ supported. 
  }
\sys_if_engine_xetex:F
  {
    \msg_fatal:nnx { ecnuthesis } { engine-unsupportd } { \c_sys_engine_str }
  }
%    \end{macrocode}
%
% \subsubsection{内部变量声明}
%
% \begin{variable}{\l_@@_tmpa_box,
%   \l_@@_tmpa_clist,
%   \l_@@_tmpb_clist,
%   \l_@@_tmpa_skip,
%   \l_@@_tmpa_dim,
%   \l_@@_tmpb_dim,
%   \l_@@_tmpa_tl,
%   \l_@@_tmpb_tl}
% 声明一些临时变量。
%    \begin{macrocode}
\box_new:N   \l_@@_tmpa_box
\clist_new:N \l_@@_tmpa_clist
\clist_new:N \l_@@_tmpb_clist
\skip_new:N  \l_@@_tmpa_skip
\tl_new:N    \l_@@_tmpa_tl
\tl_new:N    \l_@@_tmpb_tl
\dim_new:N   \l_@@_tmpa_dim
\dim_new:N   \l_@@_tmpb_dim
%    \end{macrocode}
% \end{variable}
%
% \begin{variable}{\g_@@_twoside_bool}
% 是否开启双面打印模式(默认关闭)。
%    \begin{macrocode}
\bool_new:N \g_@@_twoside_bool
\bool_set_false:N \g_@@_twoside_bool
%    \end{macrocode}
% \end{variable}
%
% \begin{variable}{\g_@@_decl_page_bool}
% 是否自动生成声明页(默认不自动生成)。
%    \begin{macrocode}
\bool_new:N \g_@@_decl_page_bool
\bool_set_false:N \g_@@_decl_page_bool
%    \end{macrocode}
% \end{variable}
%
% \begin{variable}{\g_@@_to_ctexart_clist}
% 保存由 \cls{ECNUThesis-Undergraduate} 传入 \cls{ctexrep} 宏包
% 的选项列表。
%    \begin{macrocode}
\clist_new:N \g_@@_to_ctexart_clist
%    \end{macrocode}
% \end{variable}
%
% \subsubsection{内部函数}
%
% \begin{macro}{\@@_quad:,\@@_qquad:}
% 等价于 \LaTeXe{} 中的 \tn{quad} 和 \tn{qquad}。
%    \begin{macrocode}
\cs_new:Npn \@@_quad:  { \skip_horizontal:n { 1 em } }
\cs_new:Npn \@@_qquad: { \skip_horizontal:n { 2 em } }
%    \end{macrocode}
% \end{macro}
%
% \begin{macro}{\@@_vspace:N,\@@_vspace:n,\@@_vspace:c}
% 类似 \LaTeXe{} 中的 \tn{vspace*}。
%    \begin{macrocode}
\cs_new_protected:Npn \@@_vspace:N #1
  {
    \dim_set_eq:NN \l_@@_tmpa_dim \prevdepth
    \hrule height \c_zero_dim
    \nobreak
    \skip_vertical:N #1
    \skip_vertical:N \c_zero_skip
    \dim_set_eq:NN \prevdepth \l_@@_tmpa_dim
  }
\cs_new_protected:Npn \@@_vspace:n #1
  {
    \skip_set:Nn \l_@@_tmpa_skip {#1}
    \@@_vspace:N \l_@@_tmpa_skip
  }
\cs_generate_variant:Nn \@@_vspace:N { c }
%    \end{macrocode}
% \end{macro}
%
% \begin{macro}{\@@_symbol:n}
% 等价于 \LaTeXe{} 中的 \tn{symbol}。
%    \begin{macrocode}
\cs_new:Npn \@@_symbol:n #1 { \tex_char:D #1 \scan_stop: }
%    \end{macrocode}
% \end{macro}
%
% \begin{macro}{\@@_arabic:n}
% 等价于 \LaTeXe{} 中的 \tn{arabic}。
%    \begin{macrocode}
\cs_new:Npn \@@_arabic:n #1
  { \int_to_arabic:v { c@ #1 } }
%    \end{macrocode}
% \end{macro}
%
% \begin{macro}{\@@_gadd_ltxhook:nn}
% 封装 \LaTeX{} 的钩子管理机制。本模板中的字体加载命令位于
% \texttt{begindocument/\allowbreak before} 钩子中,需确保在 \pkg{xeCJK} 之前执行。
%    \begin{macrocode}
\cs_new_protected:Npn \@@_gadd_ltxhook:nn #1#2
  { \hook_gput_code:nnn {#1} { . } {#2} }
\hook_gset_rule:nnnn { begindocument/before } { . } { < } { xeCJK }
%    \end{macrocode}
% \end{macro}
%
% \begin{macro}{\@@_patch_cmd:Nnn,\@@_preto_cmd:Nnn,\@@_appto_cmd:Nn}
% 补丁工具,来自 \pkg{ctexpatch} 宏包。
%    \begin{macrocode}
\cs_new_protected:Npn \@@_patch_cmd:Nnn #1#2#3
  {
    \ctex_patch_cmd_once:NnnnTF #1 { } {#2} {#3}
      { } { \ctex_patch_failure:N #1 }
  }
\cs_new_protected:Npn \@@_preto_cmd:Nn #1#2
  {
    \ctex_preto_cmd:NnnTF #1 { } {#2}
      { } { \ctex_patch_failure:N #1 }
  }
\cs_new_protected:Npn \@@_appto_cmd:Nn #1#2
  {
    \ctex_appto_cmd:NnnTF #1 { } {#2}
      { } { \ctex_patch_failure:N #1 }
  }
%    \end{macrocode}
% \end{macro}
%
% \begin{macro}{\@@_msg_new:nn,
%   \@@_error:n,\@@_error:nn,\@@_error:nx,\@@_error:nnn,\@@_error:nnnn,
%   \@@_warning:n,\@@_warning:nn,\@@_warning:nxx,
%   \@@_info:nx}
% 各种信息函数的缩略形式。
%    \begin{macrocode}
\cs_new:Npn \@@_msg_new:nn  { \msg_new:nnn      { ecnuthesis } }
\cs_new:Npn \@@_error:n     { \msg_error:nn     { ecnuthesis } }
\cs_new:Npn \@@_error:nn    { \msg_error:nnn    { ecnuthesis } }
\cs_new:Npn \@@_error:nx    { \msg_error:nnx    { ecnuthesis } }
\cs_new:Npn \@@_error:nnn   { \msg_error:nnnn   { ecnuthesis } }
\cs_new:Npn \@@_error:nnnn  { \msg_error:nnnnn  { ecnuthesis } }
\cs_new:Npn \@@_warning:n   { \msg_warning:nn   { ecnuthesis } }
\cs_new:Npn \@@_warning:nn  { \msg_warning:nnn  { ecnuthesis } }
\cs_new:Npn \@@_warning:nxx { \msg_warning:nnxx { ecnuthesis } }
\cs_new:Npn \@@_info:nx     { \msg_info:nnx     { ecnuthesis } }
%    \end{macrocode}
% \end{macro}
%
% \subsection{选项处理}
%
% 定义 \texttt{ecnu/option} 键值类。
%    \begin{macrocode}
\keys_define:nn { ecnu / option }
  {
%    \end{macrocode}
%
% \begin{macro}{printMode}
% 设置开启打印模式。若开启打印模式,则将输出适合双面打印的文档,否则将输出适合单面打印的
% 文档。默认关闭打印模式。
%    \begin{macrocode}
    printMode   .choice:,
    printMode / true  .code:n = 
      {
        \clist_gput_right:Nn \g_@@_to_ctexart_clist { twoside }
        \bool_set_true:N     \g_@@_twoside_bool
      },
    printMode / false .code:n = 
      {
        \clist_gput_right:Nn \g_@@_to_ctexart_clist { oneside }
        \bool_set_false:N    \g_@@_twoside_bool
      },
    printMode   .default:n = { true },
    printMode   .initial:n = { false },
%    \end{macrocode}
% \end{macro}
%
% \begin{macro}{declarePage}
% 设置开启声明页。若开启声明页,则模版将自动生成声明页。默认关闭声明页。
%    \begin{macrocode}
    declarePage   .value_forbidden:n = true,
    declarePage   .code:n =
      {
        \bool_set_true:N    \g__@@_decl_page_bool
      },
%    \end{macrocode}
% \end{macro}
%
% 处理未知选项。
%    \begin{macrocode}
    unknown .code:n = { \@@_error:n { unknown-option } }
  }
\@@_msg_new:nn { unknown-option }
  { Class~ option~ "\l_keys_key_str"~ is~ unknown. }
%    \end{macrocode}
%
% 将文档类选项传给 \texttt{ecnu/option}。
%    \begin{macrocode}
\ProcessKeysOptions { ecnu / option }
%    \end{macrocode}
%
% \subsection{载入文档类与宏包}
%
% 将选项传入 \cls{ctexrep} 文档类。
%    \begin{macrocode}
\PassOptionsToClass
  {
    UTF8,
    heading     = true,
    zihao       = -4,
    fontset     = none,
    linespread  = 1.625,
    \g_@@_to_ctexart_clist,
  }
  { ctexrep }
%    \end{macrocode}
%
% 将一些选项传入宏包。
%    \begin{macrocode}
\clist_map_inline:nn
  {
    { no-math           } { fontspec },
    { perpage           } { footmisc },
  }
  { \PassOptionsToPackage #1 }
%    \end{macrocode}
%
% 载入 \cls{ctexrep} 文档类。在使用 \XeTeX{} 编译时,\cls{ctexrep} 
% 的底层将调用 \pkg{xeCJK} 宏包以及 \pkg{fontspec} 宏包。
%    \begin{macrocode}
\LoadClass { ctexrep }
%    \end{macrocode}
%
% 载入各宏包。其中,\pkg{amsmath} 必须在 \pkg{unicode-math} 之前引入。
%    \begin{macrocode}
\RequirePackage
  {
    amsmath,
    unicode-math,
    geometry,
    fancyhdr,
    footmisc,
    graphicx,
    longtable,
    caption,
    bicaption,
    xcolor
  }
%    \end{macrocode}
%
% \subsection{页面布局}
%
% 使用 \pkg{geometry} 宏包设置纸张大小、页面边距以及页眉高度。
%
%    \begin{macrocode}
\geometry
  {
    paper     = a4paper,
    top       = 2.5 cm,
    bottom    = 2.0 cm,
    left      = 3.0 cm,
    right     = 2.5 cm,
    footskip  = 6 mm,
    headsep   = \dimexpr(9.55mm-11.7pt)
  }
%    \end{macrocode}
%
% \subsection{字体配置}
%
% \subsubsection{中西文字体配置}
%
% \begin{variable}{\g_@@_cjk_fontset_tl,\g_@@_math_fontset_tl}
% 存放字体选项值。
%    \begin{macrocode}
\tl_new:N \g_@@_cjk_fontset_tl
\tl_new:N \g_@@_math_fontset_tl
%    \end{macrocode}
% \end{variable}
%
% \begin{macro}{style/fontCJK}
% 定义中文字体选项。
%    \begin{macrocode}
\keys_define:nn { ecnu / style }
  {
    fontCJK .choices:nn =
      { fandol, mac, windows }
      { \tl_set_eq:NN \g_@@_cjk_fontset_tl \l_keys_choice_tl }
  }
%    \end{macrocode}
% \end{macro}
%
% \begin{macro}{style/fontMath}
% 定义数学字体选项。
%    \begin{macrocode}
\keys_define:nn { ecnu / style }
  {
    fontMath .choices:nn =
      { lm, times }
      { \tl_set_eq:NN \g_@@_math_fontset_tl \l_keys_choice_tl }
  }
%    \end{macrocode}
% \end{macro}
%
% \begin{macro}{
%   \@@_setmainfont:nn,
%   \@@_setsansfont:nn,
%   \@@_setmonofont:nn,
%   \@@_setmathfont:nn}
% 用于设置西文字体的辅助函数。代码来源于 \pkg{fontspec} 和 \pkg{unicode-math}。
% \begin{arguments}
%   \item 字体名
%   \item 选项
% \end{arguments}
%    \begin{macrocode}
\cs_new_protected:Npn \@@_setmainfont:nn #1#2
  { \__fontspec_main_setmainfont:nn {#2} {#1} }
\cs_new_protected:Npn \@@_setsansfont:nn #1#2
  { \__fontspec_main_setsansfont:nn {#2} {#1} }
\cs_new_protected:Npn \@@_setmonofont:nn #1#2
  { \__fontspec_main_setmonofont:nn {#2} {#1} }
\cs_new_protected:Npn \@@_setmathfont:nn #1#2
  { \__um_setmathfont:nn {#2} {#1} }
%    \end{macrocode}
% \end{macro}
%
% \begin{macro}{\@@_set_family:nnn,\@@_set_family:xnn,\@@_switch_family:n}
% 封装 CJK 字体族的设定和切换命令。
%    \begin{macrocode}
\cs_new_eq:NN \@@_set_family:nnn  \xeCJK_set_family:nnn
\cs_new_eq:NN \@@_switch_family:n \xeCJK_switch_family:n
\cs_generate_variant:Nn \@@_set_family:nnn { x }
%    \end{macrocode}
% \end{macro}
%
% \begin{macro}{
%   \@@_setCJKmainfont:nn,
%   \@@_setCJKsansfont:nn,
%   \@@_setCJKmonofont:nn}
% 用于设置中文字体的辅助函数。代码来源于 \pkg{xeCJK} 和 \pkg{ctex} 宏包。
%    \begin{macrocode}
\cs_new_protected:Npn \@@_setCJKmainfont:nn #1#2
  { \@@_set_family:xnn { \CJKrmdefault } {#2} {#1} }
\cs_new_protected:Npn \@@_setCJKsansfont:nn #1#2
  { \@@_set_family:xnn { \CJKsfdefault } {#2} {#1} }
\cs_new_protected:Npn \@@_setCJKmonofont:nn #1#2
  { \@@_set_family:xnn { \CJKttdefault } {#2} {#1} }
%    \end{macrocode}
% \end{macro}
%
% \begin{macro}{\@@_set_cjk_font_kai:nn,\ecnu@kai}
% 楷体需要单独设置。
%    \begin{macrocode}
\cs_new_protected:Npn \@@_set_cjk_font_kai:nn #1#2
  { \@@_set_family:nnn { ecnu@kai } {#2} {#1} }
\cs_new_protected:Npn \ecnu@kai
  { \@@_switch_family:n { ecnu@kai } }
%    \end{macrocode}
% \end{macro}
%
% \begin{macro}{
%   \@@_cjk_font_options:,
%   \@@_setCJKmainfont:n,
%   \@@_setCJKsansfont:n,
%   \@@_setCJKmonofont:n}
% 将 bold、italic 和 bold italic 统一按照 roman 设置。
%    \begin{macrocode}
\tl_const:Nn \@@_cjk_font_options:
 { UprightFont = *, ItalicFont = *, AutoFakeBold = true }
\cs_new_protected:Npx \@@_setCJKmainfont:n   #1
  { \@@_setCJKmainfont:nn   {#1} { \@@_cjk_font_options: } }
\cs_new_protected:Npx \@@_setCJKsansfont:n   #1
  { \@@_setCJKsansfont:nn   {#1} { \@@_cjk_font_options: } }
\cs_new_protected:Npx \@@_setCJKmonofont:n   #1
  { \@@_setCJKmonofont:nn   {#1} { \@@_cjk_font_options: } }
\cs_new_protected:Npx \@@_set_cjk_font_kai:n #1
  { \@@_set_cjk_font_kai:nn {#1} { \@@_cjk_font_options: } }
%    \end{macrocode}
% \end{macro}
%
% \begin{macro}{
%   \setmainfont,
%   \setsansfont,
%   \setmonofont,
%   \setmathfont,
%   \setCJKmainfont,
%   \setCJKsansfont,
%   \setCJKmonofont}
% 重新定义以上宏包提供的字体选择命令。
% 将其放在导言区末尾,使得用户配置不会被模板配置所覆盖。
%    \begin{macrocode}
\cs_new_protected:Npn \@@_set_font_helper:n #1
  {
    \exp_args:Nc \RenewDocumentCommand { set #1 font } { O { } m O { } }
      {
        \ctex_at_end_preamble:n
          { \use:c { @@_set #1 font:nn } {##2} { ##1, ##3 } }
      }
  }
\clist_map_inline:nn { main, sans, mono, math }
  { \@@_set_font_helper:n {#1} }
\clist_map_inline:nn { CJKmain, CJKsans, CJKmono }
  { \@@_set_font_helper:n {#1} }
%    \end{macrocode}
% \end{macro}
%
%
% \begin{macro}{\@@_load_font_times:}
% Times 系列。
%    \begin{macrocode}
\fontspec_font_if_exist:nTF { Times~ New~ Roman }
  {
    \cs_new_protected:Npn \@@_load_en_font:
      {
        \@@_setmainfont:nn { Times~ New~ Roman    } { }
        \@@_setsansfont:nn { Arial                } { }
        \@@_setmonofont:nn { Courier~ New         } { }
        \@@_setmathfont:nn { XITSMath-Regular.otf }
          { BoldFont = XITSMath-Bold.otf }
      }
  }
  {
    \cs_new_protected:Npn \@@_load_en_font:
      {
        \@@_setmainfont:nn { XITS }
          {
            Extension      = .otf,
            UprightFont    = *-Regular,
            BoldFont       = *-Bold,
            ItalicFont     = *-Italic,
            BoldItalicFont = *-BoldItalic,
          }
        \@@_setsansfont:nn { texgyreheros }
          {
            Extension      = .otf,
            UprightFont    = *-regular,
            BoldFont       = *-bold,
            ItalicFont     = *-italic,
            BoldItalicFont = *-bolditalic
          }
        \@@_setmonofont:nn { texgyrecursor }
          {
            Extension      = .otf,
            UprightFont    = *-regular,
            BoldFont       = *-bold,
            ItalicFont     = *-italic,
            BoldItalicFont = *-bolditalic,
            Ligatures      = CommonOff
          }
        \@@_setmathfont:nn { XITSMath-Regular.otf }
          { BoldFont = XITSMath-Bold.otf }
      }
  }
%    \end{macrocode}
% \end{macro}
%
% \begin{macro}{\@@_load_cjk_font_fandol:}
% Fandol 字库。注意它是安装在 TeX 发行版中的,所以使用文件名调用。
%    \begin{macrocode}
\cs_new_protected:Npn \@@_load_cjk_font_fandol:
  {
    \@@_setCJKmainfont:nn   { FandolSong }
      {
        Extension      = .otf,
        UprightFont    = *-Regular,
        BoldFont       = *-Bold,
        ItalicFont     = *-Regular,
        BoldItalicFont = *-Bold
      }
    \@@_setCJKsansfont:nn   { FandolHei  }
      {
        Extension      = .otf,
        UprightFont    = *-Regular,
        BoldFont       = *-Bold,
        ItalicFont     = *-Regular,
        BoldItalicFont = *-Bold
      }
    \@@_setCJKmonofont:nn   { FandolFang }
      {
        Extension      = .otf,
        UprightFont    = *-Regular,
        BoldFont       = *-Regular,
        ItalicFont     = *-Regular,
        BoldItalicFont = *-Regular
      }
    \@@_set_cjk_font_kai:nn { FandolKai  }
      {
        Extension      = .otf,
        UprightFont    = *-Regular,
        BoldFont       = *-Regular,
        ItalicFont     = *-Regular,
        BoldItalicFont = *-Regular
      }
  }
%    \end{macrocode}
% \end{macro}
%
% \begin{macro}{\@@_load_cjk_font_mac:}
% macOS 自带中文字体。
%    \begin{macrocode}
\cs_new_protected:Npn \@@_load_cjk_font_mac:
  {
    \@@_setCJKmainfont:nn   { STSongti-SC }
      {
        UprightFont    = *-Light,
        BoldFont       = *-Bold,
        ItalicFont     = *-Light,
        BoldItalicFont = *-Bold
      }
    \@@_setCJKsansfont:nn   { STHeitiSC   }
      {
        UprightFont    = *-Medium,
        BoldFont       = *-Medium,
        ItalicFont     = *-Medium,
        BoldItalicFont = *-Medium
      }
    \@@_setCJKmonofont:n    { STFangsong  }
    \@@_set_cjk_font_kai:nn { STKaitiSC   }
      {
        UprightFont    = *-Regular,
        BoldFont       = *-Bold,
        ItalicFont     = *-Regular,
        BoldItalicFont = *-Bold
      }
  }
%    \end{macrocode}
% \end{macro}
%
% \begin{macro}{\@@_load_cjk_font_windows:}
% Windows 自带中文字体。
%    \begin{macrocode}
\cs_new_protected:Npn \@@_load_cjk_font_windows:
  {
    \@@_setCJKmainfont:n   { SimSun   }
    \@@_setCJKsansfont:n   { SimHei   }
    \@@_setCJKmonofont:n   { FangSong }
    \@@_set_cjk_font_kai:n { KaiTi    }
  }
%    \end{macrocode}
% \end{macro}
%
% \begin{macro}{\@@_load_font:}
% 字体加载命令。
%    \begin{macrocode}
\cs_new_protected:Npn \@@_load_font:
  {
    \@@_load_en_font:
    \use:c { @@_load_cjk_font_ \g_@@_cjk_fontset_tl :   }
    \use:c { @@_load_math_font_ \g_@@_math_fontset_tl : }
  }
\ctex_at_end_preamble:n { \@@_load_font: }
%    \end{macrocode}
% \end{macro}
%
% \subsubsection{数学字体配置}
%
% 数学表达式中表示变量的拉丁字母和希腊字母均应当使用斜体。
%    \begin{macrocode}
\keys_set:nn { unicode-math }
  {
    math-style = ISO,
    bold-style = ISO,
  }
%    \end{macrocode}
%
% \subsubsection{其他配置}
%
% 声明 \cs{emph} 样式序列,使得中文下的强调命令用楷体显示。
%    \begin{macrocode}
\DeclareEmphSequence
  {
    \itshape \ecnu@kai,
    \upshape \CJKfamily { \CJKfamilydefault },
  }
%    \end{macrocode}
%
%
% \subsection{章节标题结构}
%
% 将标题层级设为五层。
%    \begin{macrocode}
\setcounter{secnumdepth}{5}
%    \end{macrocode}
%
% 设置章节的字体大小、缩进以及前后间距。对于章节标题,
% 将其所在页面的页面格式(page style)设置为 fancy。
% 由于文科模板与理科模板中四级标题与五级标题的缩进
% 不同,我们稍后再进行处理。
%    \begin{macrocode}
\keys_set:nn { ctex }
  {
    chapter = {
      format        = \sffamily\normalsize,
      pagestyle     = fancy,
      beforeskip    = 1ex,
      afterskip     = 2.5ex plus .2ex,
      fixskip       = true,
      tocline       = {\CTEXifname{\protect\numberline{\CTEXthechapter}}{}#2}
    },
    section = {
      format        = \sffamily\normalsize,
      beforeskip    = 2.5ex plus 1ex minus .2ex,
      afterskip     = 2.5ex plus .2ex,
      fixskip       = true,
    },
    subsection = {
      format        = \sffamily\normalsize,
      beforeskip    = 2.5ex plus 1ex minus .2ex,
      afterskip     = 2.5ex plus .2ex,
      indent        = 1 em,
      fixskip       = true,
    },
    subsubsection   = {
      format        = \sffamily\normalsize,
      beforeskip    = 2.5ex plus 1ex minus .2ex,
      afterskip     = 2.5ex plus .2ex,
      indent        = 2 em,
      fixskip       = true,
    },
    paragraph = {
      format        = \sffamily\normalsize,
      beforeskip    = 2ex plus 1ex minus .2ex,
      afterskip     = 1 em,
      
      fixskip       = true,
    },
  }
%    \end{macrocode}
%
% \begin{macro}{style/numbering}
% 定义 \texttt{style/numbering} 选项。
%    \begin{macrocode}
\keys_define:nn { ecnu / style }
  {
    numbering .choice:,
%    \end{macrocode}
% \end{macro}
%
% \begin{macro}{\@@_set_numbering_arabic:}
% 设置理科章节标题。
%    \begin{macrocode}
    numbering / arabic  .code:n = 
      {
        \keys_set:nn { ctex }
          {
            chapter = {
              name          = {,、},
              number        = {\arabic{chapter}},
              numberformat  = \rmfamily,
              aftername     = ,
            },
            section = {
              name          = {,},
              number        = {\arabic{chapter}.\arabic{section}},
              numberformat  = \rmfamily,
            },
            subsection = {
              name          = {,},
              number        = 
                {\arabic{chapter}.\arabic{section}.\arabic{subsection}},
              numberformat  = \rmfamily,
            }, 
            subsubsection = {
              name          = {,},
              number        = 
                {
                  \arabic{chapter}.\arabic{section}.
                  \arabic{subsection}.\arabic{subsubsection}
                },
              numberformat  = \rmfamily,
              indent        = 2 em,
            }, 
            paragraph = {
              name          = {,)},
              number        = {\arabic{paragraph}},
              numberformat  = \rmfamily,
              aftername     = ,
              indent        = 3 em,
            }
          }
      },
%    \end{macrocode}
% \end{macro}
%
% \begin{macro}{\@@_set_numbering_chinese:}
% 设置文科章节标题。
%    \begin{macrocode}
    numbering / chinese .code:n = 
      {
        \keys_set:nn { ctex }
          {
            chapter = {
              name    = {,、},
              number  = {\chinese{chapter}},
            },
            section = {
              name      = {(,)},
              number    = {\chinese{section}},
              aftername = ,
            }, 
            subsection = {
              name          = {,.},
              number        = {\arabic{subsection}},
              numberformat  = \rmfamily,
            }, 
            subsubsection = {
              name          = {(,)},
              number        = {\arabic{subsubsection}},
              numberformat  = \rmfamily,
              aftername     = \ ,
              indent        = 1 em,
            }, 
            paragraph = {
              name          = {,)},
              number        = \arabic{paragraph},
              numberformat  = \rmfamily,
              aftername     =  ,
              indent        = 2 em,
            }
          }
      },
%    \end{macrocode}
% \end{macro}
%
% 处理未知选项。
%    \begin{macrocode}
    numbering / unknown   .code:n = { \@@_error:n { unknown-value } }
  }
\@@_msg_new:nn { unknown-value }
  { The~ value~ "\l_keys_value_tl"~ is~ unknown. }
%    \end{macrocode}
%
% \begin{macro}{\@@_title:n}
% 手动生成章的标题,用于摘要、参考文献等。这些标题的字体大小
% 为小三,行距为 1.5 倍行距。
%    \begin{macrocode}
\cs_new_protected:Npn \@@_title:n #1
  {
    \group_begin:
      \ctexset 
        { 
          chapter / numbering = false,
          chapter / format    = \sffamily\bfseries\centering,
          chapter / titleformat = \large,
          chapter / afterskip = 3.5ex plus 0.5ex,
        }
      \chapter {#1}
    \group_end:
  }
\cs_generate_variant:Nn \@@_title:n { V }
%    \end{macrocode}
% \end{macro}
%
% \begin{macro}{\@@_abs_title:n}
% 手动生成中英文摘要页的标题。这些标题的字体大小
% 为小三,行距为单倍行距。
%    \begin{macrocode}
\cs_new_protected:Npn \@@_abs_title:n #1
  {
    \group_begin:
      \ctexset 
        { 
          chapter / numbering   = false,
          chapter / format      = \sffamily\bfseries\centering,
          chapter / titleformat = \large,
          chapter / beforeskip  = 2.25 bp,
          chapter / afterskip   = 2.25 bp,
        }
      \chapter* {#1}
    \group_end:
  }
\cs_generate_variant:Nn \@@_abs_title:n { V }
%    \end{macrocode}
% \end{macro}
%
% \begin{macro}{\@@_toc_title:n}
% 手动生成目录页的标题。该标题的字体大小
% 为小四,行距为单倍行距。
%    \begin{macrocode}
\cs_new_protected:Npn \@@_toc_title:n #1
  {
    \group_begin:
      \ctexset 
        { 
          chapter / numbering   = false,
          chapter / pagestyle   = empty,
          chapter / format      = \rmfamily\bfseries\centering,
          chapter / titleformat = \normalsize,
        }
      \chapter* {#1}
    \group_end:
  }
\cs_generate_variant:Nn \@@_toc_title:n { V }
%    \end{macrocode}
% \end{macro}
%
% \subsection{页眉页脚设置}
%
% 清除默认页眉页脚格式。
%    \begin{macrocode}
\fancyhf { }
%    \end{macrocode}
%
% 页眉样式。
%    \begin{macrocode}
\fancyhead[C]   { \footnotesize \sffamily \l_@@_info_title_tl }
%    \end{macrocode}
%
% 页脚样式。
%    \begin{macrocode}
\fancyfoot[C]   { \footnotesize \thepage }
%    \end{macrocode}
%
%
% \begin{macro}{\cleardoublepage}
% 重定义 \tn{cleardoublepage},使得偶数页面在没有内容时也不显示页眉页脚,见
%    \begin{macrocode}
\RenewDocumentCommand \cleardoublepage { }
  {
    \clearpage
    \bool_if:NT \g_@@_twoside_bool
      {
        \int_if_odd:nF \c@page
          { \hbox:n { } \thispagestyle { empty } \newpage }
      }
  }
%    \end{macrocode}
% \end{macro}
%
% 由于 \pkg{ctex} 宏包使用 \texttt{heading} 选项后,会将页面格式设置为 \texttt{headings}。
% 因此需要在调用 \pkg{ctex} 后重新将 \cs{pagestyle} 设置为 \texttt{fancy}。
%    \begin{macrocode}
\pagestyle { fancy }
%    \end{macrocode}
%
% \subsection{脚注}
%
% \begin{macro}{\@makefntext}
% 重定义内部脚注文字命令,使生成的脚注无缩进。
%    \begin{macrocode}
\cs_set:Npn \@makefntext #1
  {
    \mode_leave_vertical:
    \textsuperscript{\@thefnmark} #1
  }
%    \end{macrocode}
% \end{macro}
%
% \subsection{图表绘制;浮动体}
%
% 设置浮动体 figure 和 table 的标题样式。
%    \begin{macrocode}
\captionsetup
  {
    font          = small,
    labelsep      = quad,
    justification = centering
  }
%    \end{macrocode}
%
% 设置英语标题的名称。
%    \begin{macrocode}
\captionsetup [ figure ] [ bi-second ] { name=Figure }
\captionsetup [ table  ] [ bi-second ] { name=Table  }
%    \end{macrocode}
%
% 
% \begin{macro}{\@makefntext}
% 重定义图表编号。
%    \begin{macrocode}
\cs_set:Npn \thefigure
  {  \thechapter - \int_to_arabic:n { \c@figure } }
\cs_set:Npn \thetable
  {  \thechapter - \int_to_arabic:n { \c@table  } }
%    \end{macrocode}
% \end{macro}
%
% \subsection{封面}
%
% \subsubsection{信息录入}
%
% \begin{variable}{\l_@@_info_title_tl,
%   \l_@@_info_titleEN_tl,
%   \l_@@_info_author_tl,
%   \l_@@_info_studentID_tl,
%   \l_@@_info_department_tl,
%   \l_@@_info_major_tl,
%   \l_@@_info_supervisor_tl,
%   \l_@@_info_academicTitle_tl,
%   \l_@@_info_keywords_clist,
%   \l_@@_info_keywordsEN_clist}
% 封面所需的一些字段(中英文标题、作者姓名、学号、院系、专业、
% 导师姓名、导师职称、中英文关键字)。
%    \begin{macrocode}
\clist_map_inline:nn
  {
    title, titleEN, author, studentID, department, major, supervisor
    academicTitle
  }
  { \tl_new:c { l_@@_info_ #1 _tl } }
\clist_new:N \l_@@_info_keywords_clist
\clist_new:N \l_@@_info_keywordsEN_clist
%    \end{macrocode}
% \end{variable}
%
% \begin{variable}{\l_@@_info_year_int,
%   \l_@@_info_month_int,
%   \l_@@_info_graduation_year_int}
% 论文完成年份、月份与毕业年份。
%    \begin{macrocode}
\int_new:N   \l_@@_info_year_int
\int_new:N   \l_@@_info_month_int
\int_new:N   \l_@@_info_graduation_year_int
%    \end{macrocode}
% \end{variable}
%
%
% 定义 \texttt{ecnu/info} 键值类。
%    \begin{macrocode}
\keys_define:nn { ecnu / info }
  {
%    \end{macrocode}
%
% \begin{macro}{info/title,info/titleEN}
% 论文中英文标题。
%    \begin{macrocode}
    title           .tl_set:N     = \l_@@_info_title_tl,
    titleEN         .tl_set:N     = \l_@@_info_titleEN_tl,
%    \end{macrocode}
% \end{macro}
%
% \begin{macro}{info/author}
% 论文作者姓名。
%    \begin{macrocode}
    author          .tl_set:N     = \l_@@_info_author_tl,
%    \end{macrocode}
% \end{macro}
%
% \begin{macro}{info/studentID}
% 论文作者学号。
%    \begin{macrocode}
    studentID       .tl_set:N     = \l_@@_info_studentID_tl,
%    \end{macrocode}
% \end{macro}
%
% \begin{macro}{info/department}
% 论文作者所在院系。
%    \begin{macrocode}
    department      .tl_set:N     = \l_@@_info_department_tl,
%    \end{macrocode}
% \end{macro}
%
% \begin{macro}{info/major}
% 论文作者就读专业。
%    \begin{macrocode}
    major           .tl_set:N     = \l_@@_info_major_tl,
%    \end{macrocode}
% \end{macro}
%
% \begin{macro}{info/supervisor}
% 论文指导教师姓名。
%    \begin{macrocode}
    supervisor      .tl_set:N     = \l_@@_info_supervisor_tl,
%    \end{macrocode}
% \end{macro}
%
% \begin{macro}{info/academicTitle}
% 论文指导教师职称。
%    \begin{macrocode}
    academicTitle   .tl_set:N     = \l_@@_info_academicTitle_tl,
%    \end{macrocode}
% \end{macro}
%
% \begin{macro}{info/keywords,info/keywordsEN}
% 论文中英文关键词。
%    \begin{macrocode}
    keywords        .clist_set:N  = \l_@@_info_keywords_clist,
    keywordsEN      .clist_set:N  = \l_@@_info_keywordsEN_clist,
%    \end{macrocode}
% \end{macro}
%
% \begin{macro}{info/year, info/month}
% 论文完成年份与月份。
%    \begin{macrocode}
    year            .int_set:N    = \l_@@_info_year_int,
    month           .int_set:N    = \l_@@_info_month_int,
%    \end{macrocode}
% \end{macro}
%
% \begin{macro}{info/graduationYear}
% 论文作者毕业年份。
%    \begin{macrocode}
    graduationYear  .int_set:N    = \l_@@_info_graduation_year_int,
  }
%    \end{macrocode}
% \end{macro}
%
% \subsubsection{定义内部函数}
%
% \begin{macro}{\@@_spread_box:nn,\@@_spread_box:no}
% 分散对齐的水平盒子。
% \begin{arguments}
%   \item 宽度
%   \item 内容
% \end{arguments}
% 利用 \cs{tl_map_inline:nn} 在字符间插入 \texttt{hfil};紧随其后的 \tn{unskip}
% 将会去掉最后一个 \texttt{hfil}。
%    \begin{macrocode}
\cs_new_protected:Npn \@@_spread_box:nn #1#2
  {
    \mode_leave_vertical:
    \hbox_to_wd:nn {#1} { \tl_map_inline:nn {#2} { ##1 \hfil } \unskip }
  }
\cs_generate_variant:Nn \@@_spread_box:nn { no }
%    \end{macrocode}
% \end{macro}
%
% \begin{macro}{\@@_center_box:nn,\@@_center_box:Vn}
% 居中对齐的水平盒子。
%    \begin{macrocode}
\cs_new_protected:Npn \@@_center_box:nn #1#2
  {
    \mode_leave_vertical:
    \hbox_to_wd:nn {#1} { \hfil #2 \hfil }
  }
\cs_generate_variant:Nn \@@_center_box:nn { Vn }
%    \end{macrocode}
% \end{macro}
%
% \begin{macro}{\@@_fixed_width_box:nn}
% 限宽盒子(允许换行)。
%    \begin{macrocode}
\cs_new:Npn \@@_fixed_width_box:nn #1#2
  { \parbox {#1} {#2} }
%    \end{macrocode}
% \end{macro}
%
% \begin{macro}{\@@_fixed_width_center_box:nn}
% 居中对齐的限宽盒子(允许换行)。
%    \begin{macrocode}
\cs_new:Npn \@@_fixed_width_center_box:nn #1#2
  { \parbox {#1} { \centering #2 } }
%    \end{macrocode}
% \end{macro}
%
% \begin{macro}{\@@_get_text_width:Nn,\@@_get_text_width:NV}
% 获取文本宽度,并存入 \texttt{dim} 型变量。
% \begin{arguments}
%   \item \texttt{dim} 型变量
%   \item 内容
% \end{arguments}
%    \begin{macrocode}
\cs_new:Npn \@@_get_text_width:Nn #1#2
  {
    \hbox_set:Nn \l_@@_tmpa_box {#2}
    \dim_set:Nn #1 { \box_wd:N \l_@@_tmpa_box }
  }
\cs_generate_variant:Nn \@@_get_text_width:Nn { NV }
%    \end{macrocode}
% \end{macro}
%
% \begin{macro}{\@@_line_spread:N,\@@_line_spread:n}
% 设置行距。
%    \begin{macrocode}
\cs_new:Npn \@@_line_spread:N #1
  { \linespread { \fp_use:N #1 } \selectfont }
\cs_new:Npn \@@_line_spread:n #1
  { \linespread {#1} \selectfont }
%    \end{macrocode}
% \end{macro}
%
%
% \subsection{目录}
%
% 设置目录标题以及目录层级。
%    \begin{macrocode}
\tl_const:Nn \c_@@_name_toc_tl { 目录 }
\keys_set:nn { ctex }
  {
    contentsname  = \c_@@_name_toc_tl,
    tocdepth      = 1,
  }
%    \end{macrocode}
%
%
%
% 使用 \pkg{tocloft} 设置目录格式。
%    \begin{macrocode}
\PassOptionsToPackage { titles } { tocloft }
\RequirePackage { tocloft }
\RenewDocumentCommand \cftchapleader { }
  {\bfseries\cftdotfill{1.0}}
\RenewDocumentCommand \cftsecleader { }
  {\cftdotfill{1.0}}
\setlength{\cftbeforechapskip}{6pt}
\setlength{\cftchapnumwidth}{0pt}
\setlength{\cftsecnumwidth}{0pt}
\RenewDocumentCommand \@tocrmarg { } {0pt}
%    \end{macrocode}
%
% 重定义 \cs{cftchapfillnum} 和 \cs{cftsecfillnum},以去除
% 页码前的空白。
%    \begin{macrocode}
\RenewDocumentCommand \cftchapfillnum { m } 
  {
    {\cftchapleader}\nobreak
    {\cftchappagefont #1}\cftchapafterpnum\par
  }
\RenewDocumentCommand \cftsecfillnum { m }
  {
    {\cftsecleader}\nobreak
    {\cftsecpagefont #1}\cftsecafterpnum\par
  }
%    \end{macrocode}
%
% \begin{macro}{\tableofcontents}
% 修改 \cs{tableofcontents}。来自于 \LaTeXe{} 标准文档类
% \file{book.cls}
%    \begin{macrocode}
\RenewDocumentCommand \tableofcontents { }
  {
    \@@_toc_title:V \contentsname
    \group_begin:
      \@@_line_spread:n {1.2}
      \@starttoc{toc}
    \group_end:
    \cleardoublepage
  }
%    \end{macrocode}
% \end{macro}
%
% 若某章下有一个或多个小节,则目录中该章与该节中应有 6 pt 的
% 竖直间距。
%    \begin{macrocode}
\@@_preto_cmd:Nn \section
  {
    \int_compare:nNnTF { \c@section } = { \c_zero_int }
      { \addtocontents{toc}{\protect\vspace{6pt}} }
      { }
  }
%    \end{macrocode}
%
%
%
%
%
%
%
%
%
%
%
%
%
%
%
%
%
%
%
%
%
% \subsection{摘要}
%
% \begin{environment}{abstract}
% \begin{environment}{abstractEN}
% 中英文摘要环境。
%    \begin{macrocode}
\tl_const:Nn \c_@@_name_abstract_tl    { 摘要 }
\tl_const:Nn \c_@@_name_abstractEN_tl  { Abstract }
\tl_const:Nn \c_@@_name_keywords_tl    { 关键词 }
\tl_const:Nn \c_@@_name_keywordsEN_tl  { Keywords }
\tl_const:Nn \c_@@_colon_tl         { : }
\RenewDocumentEnvironment { abstract  } { }
  { \@@_abstract_begin:    } { \@@_abstract_end:      }
\NewDocumentEnvironment { abstractEN } { }
  { \@@_abstractEN_begin: } { \@@_abstractEN_end:   }
%    \end{macrocode}
% \end{environment}
% \end{environment}
%
% \begin{macro}{\@@_abstract_begin:,\@@_abstractEN_begin:}
% 摘要页标题。
%    \begin{macrocode}
\cs_new_protected:Npn \@@_abstract_begin:
  {
    % \phantomsection
    \addcontentsline{toc}{chapter}{\c_@@_name_abstract_tl}
    \@@_abs_title:V \l_@@_info_title_tl
    \small
    \hbox:n {} \mode_leave_vertical: \par \noindent
    \group_begin:
      \rmfamily\bfseries \c_@@_name_abstract_tl \c_@@_colon_tl
    \group_end:
    \par
  }
\cs_new_protected:Npn \@@_abstractEN_begin:
  {
    % \phantomsection
    \addcontentsline{toc}{chapter}{ABSTRACT}
    \@@_abs_title:V \l_@@_info_titleEN_tl
    \small
    \hbox:n {} \mode_leave_vertical: \par \noindent
    \group_begin:
      \rmfamily\bfseries \c_@@_name_abstractEN_tl \c_@@_colon_tl
    \group_end:
    \par
  }
%    \end{macrocode}
% \end{macro}
%
% \begin{macro}{\@@_abstract_end:,\@@_abstractEN_end:}
% 摘要页标题。
%    \begin{macrocode}
\cs_new_protected:Npn \@@_abstract_end:
  {
    \@@_keywords:nNn
      { \rmfamily\bfseries \c_@@_name_keywords_tl \c_@@_colon_tl }
      \l_@@_info_keywords_clist { , }
    \cleardoublepage
  }
\cs_new_protected:Npn \@@_abstractEN_end:
  {
    \_@@_keywords:nNn
      { \rmfamily\bfseries \c_@@_name_keywordsEN_tl \c_@@_colon_tl }
      \l_@@_info_keywordsEN_clist { ,~ }
    \cleardoublepage
  }
%    \end{macrocode}
% \end{macro}
%
% \begin{macro}{\@@_keywords:nNn}
%    \begin{macrocode}
\cs_new_protected:Npn \@@_keywords:nNn #1#2#3
  {
    \par \mode_leave_vertical: \par \noindent
    \@@_get_text_width:Nn \l_@@_tmpa_dim {#1}
    \group_begin: #1 \group_end:
    \parbox [t] { \dim_eval:n { \textwidth - \l_@@_tmpa_dim } }
      {
        \clist_use:Nn #2 {#3} \par
        \cs_gset:Npx \@@_keywords_prevdepth:
          { \dim_use:N \tex_prevdepth:D }
      }
  }
%    \end{macrocode}
% \end{macro}
%
%
% \subsection{附录}
%
% \begin{environment}{appendix}
% 定义附录环境。
%    \begin{macrocode}
\tl_const:Nn \c_@@_name_appendix_tl  { 附录 }
\RenewDocumentEnvironment { appendix } { }
  { \@@_appendix_begin: } { \@@_appendix_end: }

%    \end{macrocode}
% \end{environment}
%
% \begin{macro}{\@@_appendix_begin:}
% 附录标题。
%    \begin{macrocode}
\cs_new_protected:Npn \@@_appendix_begin:
  { 
    \cleardoublepage
    \@@_title:V \c_@@_name_appendix_tl
    \setcounter{chapter}{0}
    \setcounter{section}{0}
    \small
  }
%    \end{macrocode}
% \end{macro}
%
% \begin{macro}{\@@_appendix_end:}
% 在打印模式下,保证附录部分为偶数页。
%    \begin{macrocode}
\cs_new_protected:Npn \@@_appendix_end:
  {
    \cleardoublepage
  }
%    \end{macrocode}
% \end{macro}
%
%
% \subsection{致谢}
%
% \begin{environment}{acknowledgement}
% 致谢环境。
%    \begin{macrocode}
\NewDocumentEnvironment { acknowledgement } { }
  { \@@_acknowledgement_begin: }
  { \@@_acknowledgement_end: }
%    \end{macrocode}
% \end{environment}
%
% \begin{macro}{\@@_acknowledgements_begin:}
% 致谢页标题。
%    \begin{macrocode}
\tl_const:Nn \c_@@_name_acknowledgement_tl  { 致谢 }
\cs_new_protected:Npn \@@_acknowledgement_begin:
  {
    \cleardoublepage
    \@@_title:V \c_@@_name_acknowledgement_tl
    \small
  }
%    \end{macrocode}
% \end{macro}
%
% \begin{macro}{\@@_acknowledgement_begin:}
% 在打印模式下,保证致谢页为偶数页。
%    \begin{macrocode}
\cs_new_protected:Npn \@@_acknowledgement_end:
  {
    \cleardoublepage
  }
%    \end{macrocode}
% \end{macro}
%
% \subsection{参考文献}
%
% 模版使用 \cs{biblatex} 和 Biber 处理参考文献。
%
% \begin{variable}{\l_@@_bib_resource_tl}
% 声明变量。
%    \begin{macrocode}
\tl_new:N \l_@@_bib_resource_tl
%    \end{macrocode}
% \end{variable}
%
% \begin{macro}{style/fontCJK}
% 定义参考文献数据库路径。
%    \begin{macrocode}
\keys_define:nn { ecnu / style }
  {
    bibResource .tl_set:N = \l_@@_bib_resource_tl
  }
%    \end{macrocode}
% \end{macro}
%
% \pkg{biblatex} 会写入 begindocument/before 钩子
%    \begin{macrocode}
\@@_gadd_ltxhook:nn { env/document/begin }
{
  \@@_biblatex_pre_setup:
  \RequirePackage { biblatex }
  \@@_biblatex_post_setup:
}
%    \end{macrocode}
%
% \begin{macro}{\@@_biblatex_pre_setup:, \@@_biblatex_post_setup:}
% biblatex 相关设置。
%    \begin{macrocode}
\cs_new_protected:Npn \@@_biblatex_pre_setup:
  {
    \PassOptionsToPackage 
    {
      backend   = biber,
      style     = gb7714-2015,
      seconds   = true,
      hyperref  = true,
      gbpub     = false,
      gbpunctin = false,
    } 
    { biblatex }
  }
\cs_new_protected:Npn \@@_biblatex_post_setup:
  {
    \exp_args:NV \addbibresource \l_@@_bib_resource_tl
    \@@_biblatex_allow_url_break:
    \setlength{\bibitemsep}{0pt}
    \defbibheading { bibliography } [ \bibname ] 
      { \@@_title:n {##1} }
  }
%    \end{macrocode}
% \end{macro}
%
% \begin{macro}{\@@_biblatex_allow_url_break}
% \pkg{biblatex} 下允许 URL 在字母、数字和一些特殊符号处断行。
%    \begin{macrocode}
\cs_new:Npn \@@_biblatex_allow_url_break:
  {
    \int_set_eq:NN \c@biburlucpenalty  \c_one_int
    \int_set_eq:NN \c@biburlnumpenalty \c_one_int
    \int_set_eq:NN \c@biburllcpenalty  \c_one_int
  }
%    \end{macrocode}
% \end{macro}
%
% \begin{macro}{}
%    \begin{macrocode}
\NewDocumentCommand \PrintReference {} 
  {
    \printbibliography
    \cleardoublepage
  }
%    \end{macrocode}
% \end{macro}
%
%
%
% \subsection{用户接口}
%
% \begin{macro}{info,style}
% 定义元(meta)键值对。
%    \begin{macrocode}
\keys_define:nn { ecnu }
  {
    info  .meta:nn = { ecnu / info  } {#1},
    style .meta:nn = { ecnu / style } {#1}
  }
%    \end{macrocode}
% \end{macro}
%
% \begin{macro}{\ecnuSetup}
% 用户设置接口。
%    \begin{macrocode}
\NewDocumentCommand \ecnuSetup { m }
  { \keys_set:nn { ecnu } {#1} }
%    \end{macrocode}
% \end{macro}
%
% 文档类初始设置。
%    \begin{macrocode}
\keys_set:nn { ecnu / style }
  {
    fontCJK = fandol,
    fontMath = times,
    numbering = arabic,
  }
\keys_set:nn { ecnu / info }
  {
    title           = {(中文标题)},
    titleEN         = {(English Title)},
    author          = {(作者姓名)},
    studentID       = {(学号)},
    department      = {(学院)},
    major           = {(专业)},
    supervisor      = {(导师姓名)},
    academicTitle   = {(导师职称)},
    keywords        = {关键词1, 关键词2, 关键词3},
    keywordsEN      = {keyword1, keyword2, keyword3}
  }
%    \end{macrocode}
%
%
%    \begin{macrocode}
%</class>
%    \end{macrocode}
% \iffalse
%    \begin{macrocode}
%<*ecnudoc>
\ProvidesPackage{ecnudoc}
\RequirePackage{xparse}
\RequirePackage[UTF8,scheme=chinese]{ctex}
\RequirePackage{enumitem}
\RequirePackage{listings}
\RequirePackage{xcolor}
\RequirePackage[
  top     = 2.5cm,
  bottom  = 2.5cm,
  left    = 4.5cm,
  right   = 2cm,
  headsep = 3mm
]{geometry}
\setlist[enumerate]{
  topsep=0pt,
  parsep=0pt,
  itemsep=0pt
}
\setlist[itemize]{
  topsep=0pt,
  parsep=0pt,
  itemsep=0pt
}
\newlist{optdesc}{description}{3}
\setlist[optdesc]{%
  font          = \mdseries\small\ttfamily,
  align         = right,
  listparindent = \parindent,
  labelsep      = \marginparsep,
  labelindent   = -\marginparsep,
  leftmargin    = 0pt
}
\definecolor{colorRule}{RGB}{164,31,53}
\lstdefinestyle{style@base}{
  basicstyle        = \small\ttfamily,
  lineskip          = 0 pt,
  framerule         = 1 pt,
  framesep          = 3 pt,
  xleftmargin       = 4 pt,
  breaklines        = true,
  showspaces        = false,
  showstringspaces  = false,
  gobble            = 4,
  escapeinside    = {(*}{*)},
  backgroundcolor   = \color{gray!5},
  commentstyle      = \slshape\color{black!60}
}
\lstdefinestyle{style@shell}{
  style     = style@base,
  frame     = l,
  rulecolor = \color{colorRule},
  language  = bash,
}
\lstdefinestyle{style@latex}{
  style     = style@base,
  frame     = l,
  rulecolor = \color{colorRule},
  language  = [LaTeX]TeX,
}
\lstnewenvironment{latexCode}{\lstset{style=style@latex}}{}
\lstnewenvironment{shellCode}{\lstset{style=style@shell}}{}
\def\breakablethinspace{\hskip 0.16667em\relax}
\DeclareDocumentCommand\kvopt{mm}
  {\texttt{#1\breakablethinspace=\breakablethinspace#2}}
\DeclareDocumentEnvironment{arguments}{}
  {\enumerate[label={\texttt{\#\arabic*:~}}, labelsep=0pt, nolistsep]}
  {\endenumerate}
%</ecnudoc>
%    \end{macrocode}
% \fi
