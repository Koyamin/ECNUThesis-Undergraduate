% \iffalse meta-comment
% !TeX program  = XeLaTeX
% !TeX encoding = UTF-8
%
% Copyright (C) 2021--2024 by Koyamin <koyafumin@gmail.com>
%
% This work may be distributed and/or modified under the
% conditions of the LaTeX Project Public License, either
% version 1.3c of this license or (at your option) any later
% version. The latest version of this license is in:
%
%   http://www.latex-project.org/lppl.txt
%
% and version 1.3 or later is part of all distributions of
% LaTeX version 2005/12/01 or later.
% 
% \fi
%
% \iffalse
%<class|ecnudoc|ecnulogo>\NeedsTeXFormat{LaTeX2e}
%<class|ecnudoc|ecnulogo>\RequirePackage{expl3}
%<class|ecnudoc|ecnulogo>\GetIdInfo $Id: ecnuthesis.dtx 0.2.1 2024-01-20 15:00:00Z Koyamin <koyafumin@gmail.com> Qucheng <i@qcmiao.cn> $
%<class>  {Undergraduate Thesis Template for East China Normal University}
%<ecnudoc>  {Documentation class for ECNUThesis}
%<ecnulogo>  {logo for ECNUThesis}
%<class>\ProvidesExplClass{\ExplFileName}
%<ecnudoc>\ProvidesExplClass{ecnudoc}
%<ecnulogo>\ProvidesExplPackage{ecnulogo}
%<class|ecnudoc|ecnulogo>  {\ExplFileDate}{\ExplFileVersion}{\ExplFileDescription}
%<*driver>
\documentclass{ecnudoc}
\EnableCrossrefs
\CodelineIndex
\RecordChanges
\begin{document}
  \DocInput{ecnuthesis.dtx}
  \PrintIndex
\end{document}
%</driver>
% \fi
%
% \title{\cls{ECNUThesis-Undergraduate}: 华东师范大学本科生毕业论文模板}
% \author{Koyamin}
% \date{\today}
% \maketitle
% \tableofcontents
% \clearpage
%
% \begin{documentation}
%
% \section{介绍}
%
% \cls{ECNUThesis-Undergraduate} (\textbf{E}ast \textbf{C}hina 
% \textbf{N}ormal \textbf{U}niversity \LaTeX{} \textbf{Thesis} 
% Template of \textbf{Undergraduate})是华东师范大学本科生毕业
% 论文模板的一个 \LaTeX{} 实现。 \cls{ECNUThesis-Undergraduate} 旨在帮助
% 华东师范大学的本科学生使用 \LaTeX{} 完成本科生毕业论文的排版工作。
%
% \subsection{开发背景}
%
% 在本模版诞生前,YijunYuan 编写的 ECNU-Undergraduate-LaTeX 是使用人数最多、
% 流传范围最广的华东师范大学本科生毕业论文 \LaTeX{} 模版。该模版基本满足了
% 本科生毕业论文的格式要求,填补了 \LaTeX{} 版华东师范大学毕业论文模版的空白。
%
% 然而,该模版缺乏后期维护,且没有经过系统的设计,没有简明的用户接口。另一方面,
% 越来越多的华东师范大学学生选择使用 \LaTeX{} 来完成本科生毕业论文,因此
% 有必要开发一个格式规范、接口简明、易于维护的 \LaTeX{} 毕业论文模版。
%
% 本模板使用 \LaTeX3 重构了 YijunYuan 编写的 ECNU-Undergraduate-LaTeX 模板,
% 并根据《华东师范大学本科生毕业论文(设计)格式要求》的有关要求对模板进行了开发。
%
% \subsection{功能}
%
% 本模板能够帮助用户使用 \LaTeX{} 排版并生成本科生毕业论文(设计)的主要内容,包括:
% 内封面(包括学位论文诚信承诺书与使用授权说明书)、目录、中文摘要页(内含中文题名、
% 摘要、关键词)、外文摘要页(内含外文题名、摘要、关键词)、正文、致谢、参考文献及附录。
%
% \subsection{声明}
%
% 本模版是华东师范大学本科生毕业论文(设计)模板的一个第三方 \LaTeX{} 开源实现。
% 本模板的发布遵守 \LaTeX{} Project Public License (1.3.c) 协议。
% 使用本模板前,请务必阅读并同意以下事项:
%
% \begin{enumerate}
%   \item 使用本模板前,请认真阅读 \LaTeX{} Project Public License (1.3.c) 
%     协议的内容。
%   \item 本模版的开发行为未得到华东师范大学相关部门授权,本模版未经华东师范大学
%     相关部门审核。任何因使用本模版而导致的论文格式审查问题与本模版作者无关。
%   \item 任何个人或组织在修改、扩展本模板并生成新的专用模板时,请务必遵守 \LaTeX{}
%     Project Public License 协议。因违反协议而导致的任何纠纷争端与本模板作者无关。
% \end{enumerate}
%
%
% \section{简明使用教程}
%
% 本节将介绍如何使用 \cls{ECNUThesis-Undergraduate} 完成毕业论文的撰写和排版工作,
% 包括如何获取本模板、如何使用本模板提供的文档类编写论文,以及如何编译生成文档。
%
% \subsection{模版的组成}
%
% \cls{ECNUThesis-Undergraduate} 的主要组成部分由表
% \ref{table:ecnuthesis-undergraduate-components} 所示。
%
% \begin{table}[htb]
%   \caption{\cls{ECNUThesis-Undergraduate} 的主要组成部分}
%   \label{table:ecnuthesis-undergraduate-components}
%   \centering
%   \small
% \begin{tabular}{lll}
%   \hline
%  类别     & 文档类                          & 说明            \\
%   \hline
% 文档类    & \file{ecnuthesis.cls}               & 模板文档类 \cls{ecnuthesis}         \\
% 视觉形象系统 & \file{ecnu-vi-inner-cover-logo.pdf} & 内封面校徽图片       \\
% 开发文件   & ecnuthesis.dtx               & DOCSTRIP 源文件  \\
%       & ecnuthesis.ins               & DOCSTRIP 驱动文件   \\
%   \hline
% \end{tabular}
% \end{table}
%
% 其中,\file{ecnuthesis.cls} 与 \file{ecnu-vi-inner-cover-logo.pdf} 是本模板中不可缺少的两部分。
% 若缺少其中之一,则可能无法生成一份格式正确的论文。
%
%
% \subsection{获取模板}
%
% 本模板的源代码托管在 GitHub 仓库 
% \href{https://github.com/koyamin/ECNUThesis-Undergraduate}{Koyamin/ECNUThesis-Undergraduate} 中。
% 该仓库的 Release 中提供了开箱即用的模板,仅需从 Release 中下载 zip 文件即可。
% 将下载至本地的 zip 文件解压缩后,打开解压缩得到的文件夹,使用该文件夹目录中的 \texttt{thesis.tex} 来编写论文。
% 
% 本模板也提供了 \href{https://www.overleaf.com/latex/templates/ecnuthesis-latex-thesis-template-for-east-china-normal-university/szppdtkvgvpk}{Overleaf 版本},
% 打开链接并登录后即可直接编辑。
%
% \subsection{使用模板编写文档}
%
% 本小节将介绍使用本模板编写论文时的必要步骤。编写论文的步骤依次为:
% 载入文档类、配置论文基本信息与格式,以及编写正文。
%
% 在使用模板编写文档前,请确保编写论文所用到的 \TeX 源文件与 \file{ecnuthesis.cls} 
% 文件和 \file{ecnu-vi-inner-cover-logo.pdf} 处于同一目录下,且编写论文所涉及到的
% 所有源文件使用 UTF-8 编码存储。
%
% \subsubsection{载入文档类}
%
% \cls{ECNUThesis-Undergraduate} 中的 \file{ecnuthesis.cls} 文件提供了
% 一个名为 \cls{ecnuthesis} 的文档类,该文档类负责对论文进行排版。
% 在文档的第一行,我们需要使用命令 \cs{documentclass} 来载入文档类:
% 
% \begin{latexCode}
%   \documentclass{ecnuthesis}
% \end{latexCode}
%
% 在载入 \cls{ecnuthesis} 文档类时,我们还可以指定一些选项来调整论文的整体
% 排版布局,这种选项被称为模版选项。具体的模板选项以及其使用方法
% 请参阅 \ref{subsection:template-option} 节的相关内容。
% 
%
% \subsubsection{配置论文基本信息与格式}
%
% 导入 \cls{ecnuthesis} 文档类后,我们需要录入个人信息来完成论文中信息的填写。
% 此外,我们可能会需要根据具体的需要,对模版格式进行修改。我们需要在导言区使用
% \cs{ecnuSetup} 命令来录入论文所需的相关信息,以及设置论文的相关格式:
%
% \begin{latexCode}
%   \ecnuSetup {
%     style = { font = times, fontCJK = fandol },
%     info  = {
%       title       = {计算机器与智能},
%       titleEN     = {Computing Machinery and Intelligence},
%       author      = {艾伦·麦席森·图灵},
%       department  = {计算机科学与技术学院},
%       major       = {计算机科学与技术}
%     }
%   }
% \end{latexCode}
%
% \cs{ecnuSetup} 命令的具体使用方法请参阅 \ref{subsection:template-config-command} 
% 节的相关内容。
%
%
% \subsubsection{导入所需宏包}
%
% 当填写完基本信息或者完成格式设置之后,如果仍需要在编写正文的时候使用宏包,请在
% \cs{ecnuthesis} 命令之后、 \texttt{document} 环境前的导言区部分导入相应的宏包。
% 例如,若需要导入 \pkg{enumitem} 宏包,应编写以下内容
% \begin{latexCode}
%   \usepackage{enumitem}
% \end{latexCode}
%
%
% \subsubsection{编写论文}
%
% 毕业论文的编写需在 \texttt{document} 环境内进行:
% \begin{latexCode}
%   \begin{document}
%     你好,\LaTeX{}!
%   \end{document}
% \end{latexCode}
%
% 根据《华东师范大学本科生毕业论文(设计)格式要求》,毕业论文依次由以下部分构成:
% 内封面(包括学位论文诚信承诺书与使用授权说明书)、目录、中文摘要页、外文摘要页、
% 正文、致谢、参考文献及附录。接下来,将按照以上顺序简要介绍如何编写这些论文组成部分。
% 以下所有内容需依次在 \texttt{document} 环境内编写。
%
%
% \paragraph{内封面}
%
% 在使用 \texttt{document} 环境时会自动生成内封面,无需手动添加内封面。
%
%
% \paragraph{目录}
%
% 目录部分使用 \cs{tableofcontents} 命令生成。
%
%
% \paragraph{中外文摘要页}
%
% 我们将论文中的中外文摘要页部分称为论文的前置部分。编写前置部分主要分为以下步骤:
%
% \begin{enumerate}
%   \item 使用 \cs{frontmatter} 命令声明论文前置部分开始。该命令将页码调整为大写罗马字母。
%   \item 使用 \texttt{abstract} 环境编写中文摘要。该环境将自动生成中文摘要页,
%     包括中文题名、摘要、关键词。
%   \item 使用 \texttt{abstractEN} 环境以编写外文摘要。该环境将自动生成外文摘要页,
%     包括外文题名、摘要、关键词。
% \end{enumerate}
% 
% 以下是完成前置部分的一个例子。
% \begin{latexCode}
%   \frontmatter
%   \begin{abstract}
%     这是摘要。
%   \end{abstract}
%   \begin{abstractEN}
%     This is an abstract.
%   \end{abstractEN}
% \end{latexCode}
%
%
% \paragraph{正文}
%
% 编写正文部分主要分为以下步骤:
%
% \begin{enumerate}
%   \item 使用 \cs{mainmatter} 命令声明论文正文部分开始。该命令将页码调整为阿拉伯数字。
%   \item 编写正文。
%   \item 使用 \cs{backmatter} 命令声明论文正文部分结束,即后置部分开始。我们将致谢、
%     参考文献以及附录统称为后置部分。
% \end{enumerate}
%
% 以下是完成正文部分的一个例子。
% \begin{latexCode}
%   \mainmatter
%   \chapter{欢迎}
%   你好,\LaTeX{}!
%   \backmatter
% \end{latexCode}
%
% 值得注意的是,根据格式要求,正文中图表的题注应中外文对照。为此,需使用 \cs{bicaption}
% 命令代替常用的题注命令 \cs{caption}。其用法为
% \begin{latexCode}
%   \bicaption{\oarg{中文标题名}}{\oarg{外文标题名}}
% \end{latexCode}
% 例如,要为图片添加中外文题注,应如此编写:
% \begin{latexCode}
%   \begin{figure}[htb] 
%     \centering 
%       \includegraphics[width=.5\textwidth]{example-image} 
%     \bicaption{组件分布图}{Component Deploment}\label{fig-1} 
%   \end{figure}
% \end{latexCode}
%
%
% \paragraph{致谢}
%
% 致谢部分使用 \texttt{acknowledgement} 环境编写。以下是完成致谢部分的一个例子。
% \begin{latexCode}
%   \begin{acknowledgement}
%     这里是致谢。
%   \end{acknowledgement}
% \end{latexCode}
%
%
% \paragraph{参考文献}
%
% 若正文部分引用了参考文献,则可以使用命令 \cs{PrintReference} 命令生成参考文献列表。
% 以下是完成参考文献列表的一个例子。
% \begin{latexCode}
%   \PrintReference
% \end{latexCode}
%
%
% \paragraph{附录}
%
% 如有附录部分,需在参考文献后使用 \texttt{appendix} 环境编写附录。根据模板示例,
% 附录中的每一项应使用编号列表。建议使用 \texttt{enumerate} 环境编写附录内容。若要设置
% 编号列表的格式,建议使用 \pkg{enumitem} 宏包进行设置。
% 
% 以下是完成附录部分的一个例子。
%
% \begin{latexCode}
%   \begin{appendix}
%     \begin{enumerate}
%       \item 部分函数代码
%     \end{enumerate}
%   \end{appendix}
% \end{latexCode}
%
%
% \subsection{编译文档}
%
% 本模板仅支持使用 \XeLaTeX 引擎编译。为了生成正确的目录、脚注以及交叉引用,
% 您至少需要连续编译两次。
%
% 假设编写论文所使用的 \TeX 源文件名为 \file{thesis.tex}。若要编译论文,请
% 在命令行中执行
% \begin{shellCode}
%   xelatex thesis
%   biber thesis    # 若引用了参考文献, 请执行该行命令
%   xelatex thesis
% \end{shellCode}
% 或者使用 \pkg{latexmk}:
% \begin{shellCode}
%   latexmk -xelatex thesis
% \end{shellCode}
%
%
% \section{模板命令与配置项}
%
% 为了使用户更方便地使用本模板编写论文, \cls{ECNUThesis-Undergraduate} 提供
% 了一些易于使用的命令、环境和配置项,包括载入文档类时可以配置的模板选项、配置论文
% 信息与格式的导言区命令,以及编写论文时可能会使用到的命令与环境。
%
%
% \subsection{模版选项}\label{subsection:template-option}
%
% 模版选项是指在载入文档类的时候指定的选项:
% 
% \begin{latexCode}
%   \documentclass(*\oarg{模版选项}*){ecnuthesis}
% \end{latexCode}
% 
% 一些模板选项为布尔型,它们只能在 \texttt{true} 和 \texttt{false}
% 中取值。对于这些选项,\texttt{\meta{选项}=true} 中的 \texttt{=true}
% 可以被省略。
% 
% \begin{function}{printMode}
%   \begin{syntax}
%     printMode = true{\textbar}false
%   \end{syntax}
%   选择是否开启打印模式。若缺省则为关闭,反之则为开启。打印模式默认关闭。
%   若开启打印模式,则将保证内封面、中文摘要、外文摘要、目录、参考文献、附录、
%   致谢页面从奇数页开始,保证正文的最后一页为偶数页。
% 
%   开启打印模式适合双面打印纸质版论文;关闭打印模式适合提交电子版论文。
% \end{function}
%
% \begin{function}{declarePage}
%   \begin{syntax}
%     declarePage = true{\textbar}false
%   \end{syntax}
%   选择是否生成声明页。声明页默认不生成。若缺省则不生成声明页,反之则生成声明页。
%   声明页默认由模版自动生成,也可以使用 \texttt{style/declarePageResource} 选项,
%   将指定的 PDF 文件作为声明页插入到文档中。
% 
%   若选择不生成声明页,则 \texttt{style/declarePageResource} 选项将失效。
% \end{function}
%
%
% \subsection{模板配置命令}\label{subsection:template-config-command}
%
% \cls{ECNUThesis-Undergraduate} 提供了一个用于导言区的命令 \cs{ecnuSetup},用于
% 录入论文所需的相关信息,以及一些格式设置。
%
% \begin{function}{\ecnuSetup}
%   \begin{syntax}
%     \cs{ecnuSetup} = \marg{键值对列表}
%   \end{syntax}
%   该命令用于配置相关自定义选项。载入文档类后,下列所有选项均可使用该命令配置。
%   该命令需要写在导言区。
% \end{function}
%
% \cs{ecnuSetup} 的参数是一组由英文逗号(即 \texttt{,})分隔的选项列表,
% 列表中的选项通常是 \texttt{\meta{key}=\meta{value}} 的形式,我们称这种
% 形式为键值,由键值组成的选项列表称为键值列表。键值中的 \texttt{=} 周围的
% 空格不影响设置,但是键值列表之间不可以出现多余的空行。对于同一个选项,后面的
% 设置将会覆盖前一个设置。
%
% 有些选项包含子选项,比如 \texttt{style} 选项和 \texttt{info} 等选项。
% 这时候,以下两种设置方式是等价的:
% \begin{latexCode}
%   \ecnuSetup{
%     style = { font = times, fontCJK = fandol },
%     info  = {
%       title       = {计算机器与智能},
%       titleEN     = {Computing Machinery and Intelligence},
%       author      = {艾伦·麦席森·图灵},
%       department  = {计算机科学与技术学院},
%       major       = {计算机科学与技术}
%     }
%   }
% \end{latexCode}
% 或者
% \begin{latexCode}
%   \ecnuSetup{
%     style/font        = times,
%     style/fontCJK     = fandol,
%     info/title        = {计算机器与智能},
%     info/titleEN      = {Computing Machinery and Intelligence},
%     info/author       = {艾伦·麦席森·图灵},
%     info/department   = {计算机科学与技术学院},
%     info/major        = {计算机科学与技术}
%   }
% \end{latexCode}
%
% 请注意, \texttt{/} 的前后均不可以出现空白字符。
%
%
% \subsubsection{论文基本信息}
%
% 在 \cls{ECNUThesis-Undergraduate} 中,用户可以通过配置 \texttt{info} 选项
% 的子选项来完成论文基本信息的填写。本章节将介绍 \texttt{info} 选项的所有可供
% 配置的子选项。
%
% \begin{function}{info}
%   \begin{syntax}
%     info = \marg{键值对列表}
%     info/\meta{key} = \meta{value}
%   \end{syntax}
%   该选项用于录入论文信息。该选项拥有一些子选项,具体的设置子选项如下所示。
% \end{function}
%
% \begin{function}{info/title, info/titleEN}
%   \begin{syntax}
%     info/title = \marg{中文标题}
%     info/titleEN = \marg{English Title}
%   \end{syntax}
%   论文的中外文标题。该子选项用于填写内封面的中外文论文(设计)题目信息,以及中外文摘要页的标题。
% \end{function}
%% 
% \begin{function}{info/author}
%   \begin{syntax}
%     info/author = \marg{姓名}
%   \end{syntax}
%   作者的姓名。该子选项用于填写内封面的“姓名”信息。
% \end{function}
% 
% \begin{function}{info/studentID}
%   \begin{syntax}
%     info/studentID = \marg{学号}
%   \end{syntax}
%   \indent
%   作者的学号。该子选项用于填写内封面的“学号”信息。
% \end{function}
%
% \begin{function}{info/department}
%   \begin{syntax}
%     info/department = \marg{名称}
%   \end{syntax}
%   作者所在学院名称。该子选项用于填写内封面的“学院”信息。
% \end{function}
%
% \begin{function}{info/major}
%   \begin{syntax}
%     info/major = \marg{名称}
%   \end{syntax}
%   作者所在专业名称。该子选项用于填写内封面的“专业”信息。
% \end{function}
%
% \begin{function}{info/supervisor}
%   \begin{syntax}
%     info/supervisor = \marg{姓名}
%   \end{syntax}
%   论文指导教师姓名。该子选项用于填写内封面的“指导教师”信息。
% \end{function}
%
% \begin{function}{info/academicTitle}
%   \begin{syntax}
%     info/academicTitle = \marg{职称}
%   \end{syntax}
%   论文指导教师职称。该子选项用于填写内封面的“职称”信息。
% \end{function}
%
% \begin{function}{info/year}
%   \begin{syntax}
%     info/year = \meta{年份}
%   \end{syntax}
%   论文完成年份。该子选项用于填写内封面底部的年份信息。
%   若该选项被省略,模版将调用当前年份。请注意,该选项没有花括号。
% \end{function}
%
% \begin{function}{info/month}
%   \begin{syntax}
%     info/month = \meta{月份}
%   \end{syntax}
%   论文完成月份。该子选项用于填写内封面底部的月份信息。
%   若该选项被省略,模版将调用当前月份。请注意,该选项没有花括号。
% \end{function}
%
% \begin{function}{info/graduationYear}
%   \begin{syntax}
%     info/graduationYear = \meta{年份}
%   \end{syntax}
%   内封面毕业届别。该子选项用于填写内封面左上角的届别信息。
%   若该选项被省略,模版将调用 \texttt{year} 字段的值;
%   若 \texttt{year} 选项也被省略,模版将调用当前年份。
%   请注意,该选项没有花括号。
% \end{function}
%
% \begin{function}{info/keywords, info/keywordsEN}
%   \begin{syntax}
%     info/keywords = \marg{关键词}
%     info/keywordsEN = \marg{Keywords}
%   \end{syntax}
%   中外文关键词,请使用英文逗号 “\texttt{,}” 以分隔。该子选项用于填写中外文摘要页的关键词信息。
% \end{function}
%
%
% \subsubsection{论文格式设置}
%
% 在 \cls{ECNUThesis-Undergraduate} 中,用户可以通过配置 \texttt{style} 选项
% 的子选项来完成论文格式方面的相关设置。本章节将介绍 \texttt{style} 选项的所有可供
% 配置的子选项。
%
% \begin{function}{style}
%   \begin{syntax}
%     style = \marg{键值对列表}
%     style/\meta{key} = \meta{value}
%   \end{syntax}
%   该选项用于设置论文格式相关选项。其拥有一些子选项,具体的设置子选项如下所示。
% \end{function}
%
% \begin{function}{style/numbering}
%   \begin{syntax}
%     style/numbering = alpha{\textbar}\textbf{arabic}{\textbar}chinese
%   \end{syntax}
%   设置论文的章节编号样式。 \texttt{alpha} 使用数字进行编号(即外文要求),
%   \texttt{arabic} 使用数字进行编号(即理科要求),
%   \texttt{chinese} 使用数字进行编号(即文科要求),
%   默认选项为 \texttt{arabic}。
% \end{function}
%
% \begin{function}{style/fontCJK}
%   \begin{syntax}
%     style/fontCJK = \textbf{fandol}{\textbar}mac{\textbar}windows
%   \end{syntax}
%   设置论文的中文字体。默认选项为 \texttt{fandol}。具体的各配置选项及其所对应的中文字体
%   见表 \ref{table:fontCJK} 。
% \end{function}
%
% \begin{table}[ht]
%   \caption{各中文字体配置选项所对应的中文字体}
%   \label{table:fontCJK}
%   \centering
%   \small
%   \begin{tabular}{ccccc}
%   \hline
%           & 正文字体(宋体) & 无衬线字体(黑体) & 等宽字体(仿宋) & 楷体       \\
%   \hline
%   \texttt{windows} & 中易宋体     & 中易黑体      & 中易仿宋     & 中易楷体     \\
%   \texttt{mac}     & 华文宋体-简   & 华文黑体-简    & 华文仿宋     & 华文楷体-简   \\
%   \texttt{fandol}  & Fandol宋体 & Fandol黑体  & Fandol仿宋 & Fandol楷体  \\
%   \hline
%   \end{tabular}
% \end{table}
%
% \begin{function}{style/font}
%   \begin{syntax}
%     style/font = times{\textbar}\textbf{xits}{\textbar}lm
%   \end{syntax}
%   设置论文的西文字体。默认选项为 \texttt{xits}。具体的各配置选项及其所对应的西文字体
%   见表 \ref{table:font} 。
% \end{function}
%
% \begin{table}[ht]
%   \caption{各西文字体配置选项所对应的西文字体}
%   \label{table:font}
%   \centering
%   \small
%   \begin{tabular}{ccccc}
%   \hline
%       & 衬线字体            & 无衬线字体    & 等宽字体        & 数学字体      \\
%   \hline
%   \texttt{times} & Times New Roman & Arial    & Courier New & XITS Math \\
%   \texttt{xits}  & XITS            & TG Heros & TG Cursor   & XITS Math \\
%   \texttt{lm}    & LM Roman        & LM Sans  & LM Mono     & LM Math  \\
%   \hline
%   \end{tabular}
% \end{table}
%
% \begin{function}{style/bibResource}
%   \begin{syntax}
%     style/bibResource = \marg{文件}
%   \end{syntax}
%   设置参考文献数据源文件。此处需填写数据源文件所在路径(可以是绝对路径或相对路径)。
%   由于参考文献使用 biber 进行处理,故必须明确给出该数据源文件的
%   \texttt{.bib} 后缀名。
% \end{function}
%
% \begin{function}{style/logoResource}
%   \begin{syntax}
%     style/logoResource = \marg{文件}
%   \end{syntax}
%   设置封面插图数据源。此处需填写数据源文件所在路径(可以是绝对路径或相对路径)。
%   模版已自带封面插图数据源,位于 \texttt{./source/inner-cover(contains\_font).eps}。
% \end{function}
%
% \begin{function}{style/declarePageResource}
%   \begin{syntax}
%     style/declarePageResource = \marg{文件}
%   \end{syntax}
%   设置待插入的扫描版声明页 PDF 文件。若该值为空则生成模版预定义的声明页;
%   否则将指定路径所对应的 PDF 文件作为声明页插入。
% \end{function}
%
%
% \subsection{论文编写相关的命令与环境}
%
% 为符合论文格式相关要求,\cls{ECNUThesis-Undergraduate} 提供了一系列命令与环境,
% 便于用户完成论文的撰写。
%
% \subsubsection{命令}
% 
% \begin{function}{\tableofcontents}
%   \cs{tableofcontents} 命令用于生成论文目录。
% \end{function}
%
% \begin{function}{\frontmatter}
%   \cs{frontmatter} 命令用于声明论文前置部分开始。前置部分包含中外文摘要页,
%   即中文题名、摘要、关键词页以及外文题名、摘要、关键词页。
%   前置部分的页码采用大写罗马字母,并且与正文分开计数。
% \end{function}
%
% \begin{function}{\mainmatter}
%   \cs{mainmatter} 命令用于声明论文正文部分开始。正文部分是论文的核心。
%   正文部分可以分章节进行编写,也可以采用多文件编译。主体部分的页码采用阿拉伯数字。
% \end{function}
%
% \begin{function}{\backmatter}
%   \cs{backmatter} 命令用于声明论文后置部分开始。
%   后置部分包括致谢、参考文献以及附录。后置部分的页码同样使用阿拉伯数字,且与
%   正文部分相接。
% \end{function}
%
% \begin{function}{\PrintReference}
%   \cs{PrintReference} 命令用于生成参考文献列表。
% \end{function}
% 
% \subsubsection{环境}
%
% \DescribeEnv{abstract}
% \DescribeEnv{abstractEN}
% \env{abstract} 环境用于编写中(外)文摘要。使用该环境后,模板会自动生成中(外)文摘要页,
% 包括中(外)文论文标题、摘要内容以及关键词。
%
% \DescribeEnv{appendix}
% \env{appendix} 环境用于编写论文附录部分。
%
% \DescribeEnv{acknowledgement}
% \env{acknowledgement} 环境用于编写论文致谢部分。
%
%
% \section{宏集依赖情况}
%
% \cls{ECNUThesis-Undergraduate} 显式调用了以下宏包或文档类:
% \begin{itemize}
%   \item \pkg{xparse} 宏包、\pkg{xtemplate} 宏包以及 \pkg{l3keys2e} 宏包,
%     用于扩展 \LaTeX3 编程环境。属于 \pkg{l3packages} 宏集。
%   \item \cls{ctexrep} 文档类,提供中文排版的通用框架。属于 \CTeX{} 宏集。
%   \item \pkg{amsmath} 宏包,用于增强 \LaTeX 的数学排版功能。
%   \item \pkg{unicode-math} 宏包,用于处理使用 Unicode 编码的 OpenType 数学字体。
%   \item \pkg{geometry} 宏包,用于调整页面尺寸。
%   \item \pkg{fancyhdr} 宏包,用于设置页眉和页脚。
%   \item \pkg{footmisc} 宏包,用于设置脚注。
%   \item \pkg{caption} 宏包,用于设置图片与表格的题注。
%   \item \pkg{bicaption} 宏包,用于支持中外文对照版题注。
%   \item \pkg{xcolor} 宏包,用于提供彩色支持。
%   \item \pkg{xeCJKfntef} 宏包,用于实现内封面页的中文下划线。
%   \item \pkg{tocloft} 宏包,用于设置目录格式。
%   \item \pkg{biblatex} 宏包,用于参考文献的格式设置,并依赖 biber 程序。
%   \item \pkg{hyperref} 宏包,用于提供交叉引用、超链接以及电子书签等功能。
%   \item 若 \texttt{style/declarePageResource} 选项指定了需要插入至论文的 PDF 声明页
%     文件,则本模板还将调用 \pkg{pdfpages} 宏包。
% \end{itemize}
%

%
% \section{致谢}
%
%
%
%
%
%
% \end{documentation}
%
% 
%
% \begin{implementation}
%
% \iffalse
%
% \section{实现细节}
%
% \cls{ECNUThesis-Undergraduate} 使用 \LaTeX3 语法编写,
% 依赖 \pkg{expl3} 环境,并依赖 \pkg{l3packages} 中的相关宏包。
%
% 根据 \LaTeX3 的语法规则,代码中的空格、换行、回车与制表符将被完全忽略,
% 而下划线 \texttt{\_} 和冒号 \texttt{:} 则可作为一般字母使用。
%
% \subsection{准备}
%
% 
%    \begin{macrocode}
%<@@=ecnu>
%<*class>
%    \end{macrocode}
%
% 导入 \LaTeX3 编程环境。
%
%    \begin{macrocode}
\RequirePackage { xparse, xtemplate, l3keys2e }
%    \end{macrocode}
%
% 目前 \cls{ECNUThesis-Undergraduate} 仅支持 \XeTeX{}。
%    \begin{macrocode}
\msg_new:nnn { ecnuthesis } { engine-unsupportd } 
  { \\
    This~ class~ requires~ XeTeX~ to~ compile. \\
    The~ engine~ "#1"~ is~ not~ supported. 
  }
\sys_if_engine_xetex:F
  {
    \msg_fatal:nnx { ecnuthesis } { engine-unsupportd } { \c_sys_engine_str }
  }
%    \end{macrocode}
%
% \subsubsection{内部变量声明}
%
% \begin{variable}{\l_@@_tmpa_box,
%   \l_@@_tmpa_clist,
%   \l_@@_tmpb_clist,
%   \l_@@_tmpa_skip,
%   \l_@@_tmpa_dim,
%   \l_@@_tmpb_dim,
%   \l_@@_tmpa_tl,
%   \l_@@_tmpb_tl}
% 声明一些临时变量。
%    \begin{macrocode}
\box_new:N   \l_@@_tmpa_box
\clist_new:N \l_@@_tmpa_clist
\clist_new:N \l_@@_tmpb_clist
\skip_new:N  \l_@@_tmpa_skip
\tl_new:N    \l_@@_tmpa_tl
\tl_new:N    \l_@@_tmpb_tl
\dim_new:N   \l_@@_tmpa_dim
\dim_new:N   \l_@@_tmpb_dim
%    \end{macrocode}
% \end{variable}
%
% \begin{variable}{\g_@@_twoside_bool}
% 是否开启双面打印模式(默认关闭)。
%    \begin{macrocode}
\bool_new:N \g_@@_twoside_bool
\bool_set_false:N \g_@@_twoside_bool
%    \end{macrocode}
% \end{variable}
%
% \begin{variable}{\g_@@_decl_page_bool}
% 是否自动生成声明页(默认不自动生成)。
%    \begin{macrocode}
\bool_new:N \g_@@_decl_page_bool
\bool_set_false:N \g_@@_decl_page_bool
%    \end{macrocode}
% \end{variable}
%
% \begin{variable}{\g_@@_to_ctexart_clist}
% 保存由 \cls{ECNUThesis-Undergraduate} 传入 \cls{ctexrep} 宏包
% 的选项列表。
%    \begin{macrocode}
\clist_new:N \g_@@_to_ctexart_clist
%    \end{macrocode}
% \end{variable}
%
% \subsubsection{内部函数}
%
% \begin{macro}{\@@_quad:,\@@_qquad:}
% 等价于 \LaTeXe{} 中的 \tn{quad} 和 \tn{qquad}。
%    \begin{macrocode}
\cs_new:Npn \@@_quad:  { \skip_horizontal:n { 1 em } }
\cs_new:Npn \@@_qquad: { \skip_horizontal:n { 2 em } }
%    \end{macrocode}
% \end{macro}
%
% \begin{macro}{\@@_vspace:N,\@@_vspace:n,\@@_vspace:c}
% 类似 \LaTeXe{} 中的 \tn{vspace*}。
%    \begin{macrocode}
\cs_new_protected:Npn \@@_vspace:N #1
  {
    \dim_set_eq:NN \l_@@_tmpa_dim \prevdepth
    \hrule height \c_zero_dim
    \nobreak
    \skip_vertical:N #1
    \skip_vertical:N \c_zero_skip
    \dim_set_eq:NN \prevdepth \l_@@_tmpa_dim
  }
\cs_new_protected:Npn \@@_vspace:n #1
  {
    \skip_set:Nn \l_@@_tmpa_skip {#1}
    \@@_vspace:N \l_@@_tmpa_skip
  }
\cs_generate_variant:Nn \@@_vspace:N { c }
%    \end{macrocode}
% \end{macro}
%
% \begin{macro}{\@@_symbol:n}
% 等价于 \LaTeXe{} 中的 \tn{symbol}。
%    \begin{macrocode}
\cs_new:Npn \@@_symbol:n #1 { \tex_char:D #1 \scan_stop: }
%    \end{macrocode}
% \end{macro}
%
% \begin{macro}{\@@_arabic:n}
% 等价于 \LaTeXe{} 中的 \tn{arabic}。
%    \begin{macrocode}
\cs_new:Npn \@@_arabic:n #1
  { \int_to_arabic:v { c@ #1 } }
%    \end{macrocode}
% \end{macro}
%
% \begin{macro}{\@@_gadd_ltxhook:nn}
% 封装 \LaTeX{} 的钩子管理机制。本模板中的字体加载命令位于
% \texttt{begindocument/\allowbreak before} 钩子中,需确保在 \pkg{xeCJK} 之前执行。
%    \begin{macrocode}
\cs_new_protected:Npn \@@_gadd_ltxhook:nn #1#2
  { \hook_gput_code:nnn {#1} { . } {#2} }
\hook_gset_rule:nnnn { begindocument/before } { . } { < } { xeCJK }
%    \end{macrocode}
% \end{macro}
%
% \begin{macro}{\@@_patch_cmd:Nnn,\@@_preto_cmd:Nnn,\@@_appto_cmd:Nn}
% 补丁工具,来自 \pkg{ctexpatch} 宏包。
%    \begin{macrocode}
\cs_new_protected:Npn \@@_patch_cmd:Nnn #1#2#3
  {
    \ctex_patch_cmd_once:NnnnTF #1 { } {#2} {#3}
      { } { \ctex_patch_failure:N #1 }
  }
\cs_new_protected:Npn \@@_preto_cmd:Nn #1#2
  {
    \ctex_preto_cmd:NnnTF #1 { } {#2}
      { } { \ctex_patch_failure:N #1 }
  }
\cs_new_protected:Npn \@@_appto_cmd:Nn #1#2
  {
    \ctex_appto_cmd:NnnTF #1 { } {#2}
      { } { \ctex_patch_failure:N #1 }
  }
%    \end{macrocode}
% \end{macro}
%
% \begin{macro}{\@@_msg_new:nn,
%   \@@_error:n,\@@_error:nn,\@@_error:nx,\@@_error:nnn,\@@_error:nnnn,
%   \@@_warning:n,\@@_warning:nn,\@@_warning:nxx,
%   \@@_info:nx}
% 各种信息函数的缩略形式。
%    \begin{macrocode}
\cs_new:Npn \@@_msg_new:nn  { \msg_new:nnn      { ecnuthesis } }
\cs_new:Npn \@@_error:n     { \msg_error:nn     { ecnuthesis } }
\cs_new:Npn \@@_error:nn    { \msg_error:nnn    { ecnuthesis } }
\cs_new:Npn \@@_error:nx    { \msg_error:nnx    { ecnuthesis } }
\cs_new:Npn \@@_error:nnn   { \msg_error:nnnn   { ecnuthesis } }
\cs_new:Npn \@@_error:nnnn  { \msg_error:nnnnn  { ecnuthesis } }
\cs_new:Npn \@@_warning:n   { \msg_warning:nn   { ecnuthesis } }
\cs_new:Npn \@@_warning:nn  { \msg_warning:nnn  { ecnuthesis } }
\cs_new:Npn \@@_warning:nxx { \msg_warning:nnxx { ecnuthesis } }
\cs_new:Npn \@@_info:nx     { \msg_info:nnx     { ecnuthesis } }
%    \end{macrocode}
% \end{macro}
%
% \subsection{选项处理}
%
% 定义 \texttt{ecnu/option} 键值类。
%    \begin{macrocode}
\keys_define:nn { ecnu / option }
  {
%    \end{macrocode}
%
% \begin{macro}{printMode}
% 设置开启打印模式。若开启打印模式,则将输出适合双面打印的文档,否则将输出适合单面打印的
% 文档。默认关闭打印模式。
%    \begin{macrocode}
    printMode   .choice:,
    printMode / true  .code:n = 
      {
        \clist_gput_right:Nn \g_@@_to_ctexart_clist { twoside }
        \bool_set_true:N     \g_@@_twoside_bool
      },
    printMode / false .code:n = 
      {
        \clist_gput_right:Nn \g_@@_to_ctexart_clist { oneside }
        \bool_set_false:N    \g_@@_twoside_bool
      },
    printMode   .default:n = { true },
    printMode   .initial:n = { false },
%    \end{macrocode}
% \end{macro}
%
% \begin{macro}{declarePage}
% 设置开启声明页。若开启声明页,则模版将自动生成声明页。默认关闭声明页。
%    \begin{macrocode}
    declarePage   .choice:,
    declarePage / true  .code:n =
      {
        \bool_set_true:N    \g__ecnu_decl_page_bool
      },
    declarePage / false  .code:n =
      {
        \bool_set_false:N    \g__ecnu_decl_page_bool
      },
    declarePage   .default:n = { true },
    declarePage   .initial:n = { false },
%    \end{macrocode}
% \end{macro}
%
% 处理未知选项。
%    \begin{macrocode}
    unknown .code:n = { \@@_error:n { unknown-option } }
  }
\@@_msg_new:nn { unknown-option }
  { Class~ option~ "\l_keys_key_str"~ is~ unknown. }
%    \end{macrocode}
%
% 将文档类选项传给 \texttt{ecnu/option}。
%    \begin{macrocode}
\ProcessKeysOptions { ecnu / option }
%    \end{macrocode}
%
% \subsection{载入文档类与宏包}
%
% 将选项传入 \cls{ctexrep} 文档类。正文字号应设置为小四,
% 正文行间距应设置为 1.5 倍行距。由于在 Word 中,中易宋体
% 的单倍行距的长度约为字号长度的 1.3 倍,而在 \LaTeX 中
% 单倍行距的长度是字号长度的 1.2 倍,这意味着 Word 中的
% 1.5 倍行距相当于 \LaTeX 中的约 $1.3 \times 1.5 / 1.2 = 1.625$
% 倍行距。
%    \begin{macrocode}
\PassOptionsToClass
  {
    UTF8,
    heading     = true,
    zihao       = -4,
    fontset     = none,
    linespread  = 1.625,
    \g_@@_to_ctexart_clist,
  }
  { ctexrep }
%    \end{macrocode}
%
% 将一些选项传入宏包。
%    \begin{macrocode}
\clist_map_inline:nn
  {
    { no-math           } { fontspec },
    { perpage           } { footmisc },
  }
  { \PassOptionsToPackage #1 }
%    \end{macrocode}
%
% 载入 \cls{ctexrep} 文档类。在使用 \XeTeX{} 编译时,\cls{ctexrep} 
% 的底层将调用 \pkg{xeCJK} 宏包以及 \pkg{fontspec} 宏包。
%    \begin{macrocode}
\LoadClass { ctexrep }
%    \end{macrocode}
%
% 载入各宏包。其中,\pkg{amsmath} 必须在 \pkg{unicode-math} 之前引入。
%    \begin{macrocode}
\RequirePackage
  {
    amsmath,
    unicode-math,
    geometry,
    fancyhdr,
    footmisc,
    graphicx,
    caption,
    bicaption,
    xcolor,
    xeCJKfntef
  }
%    \end{macrocode}
%
% \subsection{页面布局}
%
% 使用 \pkg{geometry} 宏包设置纸张大小、页面边距以及页眉高度。
%
%    \begin{macrocode}
\geometry
  {
    paper     = a4paper,
    top       = 2.5 cm,
    bottom    = 2.0 cm,
    left      = 3.0 cm,
    right     = 2.5 cm,
    footskip  = 6 mm,
    headsep   = \dimexpr(9.55mm-11.7pt)
  }
%    \end{macrocode}
%
% \subsection{字体配置}
%
% \subsubsection{中西文字体配置}
%
% \begin{variable}{\g_@@_cjk_fontset_tl,\g_@@_fontset_tl}
% 存放字体选项值。
%    \begin{macrocode}
\tl_new:N \g_@@_cjk_fontset_tl
\tl_new:N \g_@@_fontset_tl
%    \end{macrocode}
% \end{variable}
%
% \begin{macro}{style/font}
% 定义西文字体选项。
%    \begin{macrocode}
\keys_define:nn { ecnu / style }
  {
    font .choices:nn =
      { lm, times, xits }
      { \tl_set_eq:NN \g_@@_fontset_tl \l_keys_choice_tl }
  }
%    \end{macrocode}
% \end{macro}
%
% \begin{macro}{style/fontCJK}
% 定义中文字体选项。
%    \begin{macrocode}
\keys_define:nn { ecnu / style }
  {
    fontCJK .choices:nn =
      { fandol, mac, windows }
      { \tl_set_eq:NN \g_@@_cjk_fontset_tl \l_keys_choice_tl }
  }
%    \end{macrocode}
% \end{macro}
%
% \begin{macro}{
%   \@@_setmainfont:nn,
%   \@@_setsansfont:nn,
%   \@@_setmonofont:nn,
%   \@@_setmathfont:nn}
% 用于设置西文字体的辅助函数。代码来源于 \pkg{fontspec} 和 \pkg{unicode-math}。
% \begin{arguments}
%   \item 字体名
%   \item 选项
% \end{arguments}
%    \begin{macrocode}
\cs_new_protected:Npn \@@_setmainfont:nn #1#2
  { \__fontspec_main_setmainfont:nn {#2} {#1} }
\cs_new_protected:Npn \@@_setsansfont:nn #1#2
  { \__fontspec_main_setsansfont:nn {#2} {#1} }
\cs_new_protected:Npn \@@_setmonofont:nn #1#2
  { \__fontspec_main_setmonofont:nn {#2} {#1} }
\cs_new_protected:Npn \@@_setmathfont:nn #1#2
  { \__um_setmathfont:nn {#2} {#1} }
%    \end{macrocode}
% \end{macro}
%
% \begin{macro}{\@@_set_family:nnn,\@@_set_family:xnn,\@@_switch_family:n}
% 封装 CJK 字体族的设定和切换命令。
%    \begin{macrocode}
\cs_new_eq:NN \@@_set_family:nnn  \xeCJK_set_family:nnn
\cs_new_eq:NN \@@_switch_family:n \xeCJK_switch_family:n
\cs_generate_variant:Nn \@@_set_family:nnn { x }
%    \end{macrocode}
% \end{macro}
%
% \begin{macro}{
%   \@@_setCJKmainfont:nn,
%   \@@_setCJKsansfont:nn,
%   \@@_setCJKmonofont:nn}
% 用于设置中文字体的辅助函数。代码来源于 \pkg{xeCJK} 和 \pkg{ctex} 宏包。
%    \begin{macrocode}
\cs_new_protected:Npn \@@_setCJKmainfont:nn #1#2
  { \@@_set_family:xnn { \CJKrmdefault } {#2} {#1} }
\cs_new_protected:Npn \@@_setCJKsansfont:nn #1#2
  { \@@_set_family:xnn { \CJKsfdefault } {#2} {#1} }
\cs_new_protected:Npn \@@_setCJKmonofont:nn #1#2
  { \@@_set_family:xnn { \CJKttdefault } {#2} {#1} }
%    \end{macrocode}
% \end{macro}
%
% \begin{macro}{
%   \@@_set_cjk_font_kai:nn,
%   \ecnu@kai}
% 楷体需要单独设置。
%    \begin{macrocode}
\cs_new_protected:Npn \@@_set_cjk_font_kai:nn #1#2
  { \@@_set_family:nnn { ecnu@kai } {#2} {#1} }
\cs_new_protected:Npn \ecnu@kai
  { \@@_switch_family:n { ecnu@kai } }
%    \end{macrocode}
% \end{macro}
%
% \begin{macro}{
%   \@@_cjk_font_options:,
%   \@@_setCJKmainfont:n,
%   \@@_setCJKsansfont:n,
%   \@@_setCJKmonofont:n}
% 将 bold、italic 和 bold italic 统一按照 roman 设置。
%    \begin{macrocode}
\tl_const:Nn \@@_cjk_font_options:
 { UprightFont = *, ItalicFont = *, AutoFakeBold = true }
\cs_new_protected:Npx \@@_setCJKmainfont:n   #1
  { \@@_setCJKmainfont:nn   {#1} { \@@_cjk_font_options: } }
\cs_new_protected:Npx \@@_setCJKsansfont:n   #1
  { \@@_setCJKsansfont:nn   {#1} { \@@_cjk_font_options: } }
\cs_new_protected:Npx \@@_setCJKmonofont:n   #1
  { \@@_setCJKmonofont:nn   {#1} { \@@_cjk_font_options: } }
\cs_new_protected:Npx \@@_set_cjk_font_kai:n #1
  { \@@_set_cjk_font_kai:nn {#1} { \@@_cjk_font_options: } }
%    \end{macrocode}
% \end{macro}
%
% \begin{macro}{
%   \setmainfont,
%   \setsansfont,
%   \setmonofont,
%   \setmathfont,
%   \setCJKmainfont,
%   \setCJKsansfont,
%   \setCJKmonofont}
% 重新定义以上宏包提供的字体选择命令。
% 将其放在导言区末尾,使得用户配置不会被模板配置所覆盖。
%    \begin{macrocode}
\cs_new_protected:Npn \@@_set_font_helper:n #1
  {
    \exp_args:Nc \RenewDocumentCommand { set #1 font } { O { } m O { } }
      {
        \ctex_at_end_preamble:n
          { \use:c { @@_set #1 font:nn } {##2} { ##1, ##3 } }
      }
  }
\clist_map_inline:nn { main, sans, mono, math }
  { \@@_set_font_helper:n {#1} }
\clist_map_inline:nn { CJKmain, CJKsans, CJKmono }
  { \@@_set_font_helper:n {#1} }
%    \end{macrocode}
% \end{macro}
%
%
% \begin{macro}{\@@_load_font_times:}
% Times 系列西文字体。
%    \begin{macrocode}
\cs_new_protected:Npn \@@_load_font_times:
  {
    \@@_setmainfont:nn { Times~ New~ Roman    } { }
    \@@_setsansfont:nn { Arial                } { }
    \@@_setmonofont:nn { Courier~ New         } { }
    \@@_setmathfont:nn { XITSMath-Regular.otf }
      { BoldFont = XITSMath-Bold.otf }
  }
%    \end{macrocode}
% \end{macro}
%
% \begin{macro}{\@@_load_font_xits:}
% XITS 系列西文字体。
%    \begin{macrocode}
\cs_new_protected:Npn \@@_load_font_xits:
  {
    \@@_setmainfont:nn { XITS }
      {
        Extension      = .otf,
        UprightFont    = *-Regular,
        BoldFont       = *-Bold,
        ItalicFont     = *-Italic,
        BoldItalicFont = *-BoldItalic,
      }
    \@@_setsansfont:nn { texgyreheros }
      {
        Extension      = .otf,
        UprightFont    = *-regular,
        BoldFont       = *-bold,
        ItalicFont     = *-italic,
        BoldItalicFont = *-bolditalic
      }
    \@@_setmonofont:nn { texgyrecursor }
      {
        Extension      = .otf,
        UprightFont    = *-regular,
        BoldFont       = *-bold,
        ItalicFont     = *-italic,
        BoldItalicFont = *-bolditalic,
        Ligatures      = CommonOff
      }
    \@@_setmathfont:nn { XITSMath-Regular.otf }
      { BoldFont = XITSMath-Bold.otf }
  }
%    \end{macrocode}
% \end{macro}
%
% \begin{macro}{\@@_load_font_xits:}
% Latin Modern 系列西文字体。由于已作为默认字体,所以仅需额外处理数学部分。
%    \begin{macrocode}
\cs_new_protected:Npn \@@_load_font_lm:
  { \@@_setmathfont:nn { latinmodern-math.otf } { } }
%    \end{macrocode}
% \end{macro}
%
% \begin{macro}{\@@_load_cjk_font_fandol:}
% Fandol 字库中文字体。由于其安装在 TeX 发行版中,故使用文件名调用。
%    \begin{macrocode}
\cs_new_protected:Npn \@@_load_cjk_font_fandol:
  {
    \@@_setCJKmainfont:nn   { FandolSong }
      {
        Extension      = .otf,
        UprightFont    = *-Regular,
        BoldFont       = *-Bold,
        ItalicFont     = *-Regular,
        BoldItalicFont = *-Bold
      }
    \@@_setCJKsansfont:nn   { FandolHei  }
      {
        Extension      = .otf,
        UprightFont    = *-Regular,
        BoldFont       = *-Bold,
        ItalicFont     = *-Regular,
        BoldItalicFont = *-Bold
      }
    \@@_setCJKmonofont:nn   { FandolFang }
      {
        Extension      = .otf,
        UprightFont    = *-Regular,
        BoldFont       = *-Regular,
        ItalicFont     = *-Regular,
        BoldItalicFont = *-Regular
      }
    \@@_set_cjk_font_kai:nn { FandolKai  }
      {
        Extension      = .otf,
        UprightFont    = *-Regular,
        BoldFont       = *-Regular,
        ItalicFont     = *-Regular,
        BoldItalicFont = *-Regular
      }
  }
%    \end{macrocode}
% \end{macro}
%
% \begin{macro}{\@@_load_cjk_font_mac:}
% macOS 自带的中文字体。
%    \begin{macrocode}
\cs_new_protected:Npn \@@_load_cjk_font_mac:
  {
    \@@_setCJKmainfont:nn   { STSongti-SC }
      {
        UprightFont    = *-Light,
        BoldFont       = *-Bold,
        ItalicFont     = *-Light,
        BoldItalicFont = *-Bold
      }
    \@@_setCJKsansfont:nn   { STHeitiSC   }
      {
        UprightFont    = *-Medium,
        BoldFont       = *-Medium,
        ItalicFont     = *-Medium,
        BoldItalicFont = *-Medium
      }
    \@@_setCJKmonofont:n    { STFangsong  }
    \@@_set_cjk_font_kai:nn { STKaitiSC   }
      {
        UprightFont    = *-Regular,
        BoldFont       = *-Bold,
        ItalicFont     = *-Regular,
        BoldItalicFont = *-Bold
      }
  }
%    \end{macrocode}
% \end{macro}
%
% \begin{macro}{\@@_load_cjk_font_windows:}
% Windows 自带的中文字体。
%    \begin{macrocode}
\cs_new_protected:Npn \@@_load_cjk_font_windows:
  {
    \@@_setCJKmainfont:n   { SimSun   }
    \@@_setCJKsansfont:n   { SimHei   }
    \@@_setCJKmonofont:n   { FangSong }
    \@@_set_cjk_font_kai:n { KaiTi    }
  }
%    \end{macrocode}
% \end{macro}
%
% \begin{macro}{\@@_load_font:}
% 字体加载命令。
%    \begin{macrocode}
\cs_new_protected:Npn \@@_load_font:
  {
    \use:c { @@_load_font_ \g_@@_fontset_tl :         }
    \use:c { @@_load_cjk_font_ \g_@@_cjk_fontset_tl : }
  }
\ctex_at_end_preamble:n { \@@_load_font: }
%    \end{macrocode}
% \end{macro}
%
% \subsubsection{数学字体配置}
%
% 数学表达式中表示变量的拉丁字母和希腊字母均应当使用斜体。
%    \begin{macrocode}
\keys_set:nn { unicode-math }
  {
    math-style = ISO,
    bold-style = ISO,
  }
%    \end{macrocode}
%
% \subsubsection{其他配置}
%
% 声明 \cs{emph} 样式序列,使得中文下的强调命令用楷体显示。
%    \begin{macrocode}
\DeclareEmphSequence
  {
    \itshape \ecnu@kai,
    \upshape \CJKfamily { \CJKfamilydefault },
  }
%    \end{macrocode}
%
%
% \subsection{章节标题结构}
%
% 将标题层级设为五层。
%    \begin{macrocode}
\setcounter{secnumdepth}{5}
%    \end{macrocode}
%
% 设置章节的字体大小、缩进以及前后间距。毕业论文正文层次不
% 超过 5 层次。由于文科模板与理科模板中四级标题与五级标题的
% 缩进不同,我们稍后再进行处理。
%    \begin{macrocode}
\keys_set:nn { ctex }
  {
    chapter = {
      format        = \sffamily\normalsize,
      pagestyle     = fancy,
      beforeskip    = 1ex,
      afterskip     = 2.5ex plus .2ex,
      fixskip       = true,
      tocline       = {\CTEXifname{\protect\numberline{\CTEXthechapter}}{}#2}
    },
    section = {
      format        = \sffamily\normalsize,
      beforeskip    = 2.5ex plus 1ex minus .2ex,
      afterskip     = 2.5ex plus .2ex,
      fixskip       = true,
    },
    subsection = {
      format        = \sffamily\normalsize,
      beforeskip    = 2.5ex plus 1ex minus .2ex,
      afterskip     = 2.5ex plus .2ex,
      indent        = 1 em,
      fixskip       = true,
    },
    subsubsection   = {
      format        = \sffamily\normalsize,
      beforeskip    = 2.5ex plus 1ex minus .2ex,
      afterskip     = 2.5ex plus .2ex,
      indent        = 2 em,
      fixskip       = true,
    },
    paragraph = {
      format        = \sffamily\normalsize,
      beforeskip    = 2.5ex plus 1ex minus .2ex,
      afterskip     = 2.5ex plus .2ex,
      runin         = false,
      fixskip       = true,
    },
  }
%    \end{macrocode}
%
% \begin{macro}{style/numbering}
% 定义 \texttt{style/numbering} 选项。
%    \begin{macrocode}
\keys_define:nn { ecnu / style }
  {
    numbering .choice:,
%    \end{macrocode}
% \end{macro}
%
% 设置理科模板的章节标题。理科模板的四级标题前的缩进为 2 个字符;
% 五级标题前的缩进为 3 个字符。
%    \begin{macrocode}
    numbering / arabic  .code:n = 
      {
        \keys_set:nn { ctex }
          {
            chapter = {
              name          = {,、},
              number        = {\arabic{chapter}},
              numberformat  = \rmfamily,
              aftername     = ,
            },
            section = {
              name          = {,},
              number        = {\arabic{chapter}.\arabic{section}},
              numberformat  = \rmfamily,
            },
            subsection = {
              name          = {,},
              number        = 
                {\arabic{chapter}.\arabic{section}.\arabic{subsection}},
              numberformat  = \rmfamily,
            }, 
            subsubsection = {
              name          = {,},
              number        = 
                {
                  \arabic{chapter}.\arabic{section}.
                  \arabic{subsection}.\arabic{subsubsection}
                },
              numberformat  = \rmfamily,
              indent        = 2 em,
            }, 
            paragraph = {
              name          = {,)},
              number        = {\arabic{paragraph}},
              numberformat  = \rmfamily,
              aftername     = ,
              indent        = 3 em,
            }
          }
      },
%    \end{macrocode}
%
% 设置文科模板章节标题。文科模板的四级标题前的缩进为 2 个字符;
% 五级标题前的缩进为 4 个字符。
%    \begin{macrocode}
    numbering / chinese .code:n = 
      {
        \keys_set:nn { ctex }
          {
            chapter = {
              name          = {,、},
              number        = {\chinese{chapter}},
              numberformat  = \sffamily,
            },
            section = {
              name          = {(,)},
              number        = {\chinese{section}},
              numberformat  = \sffamily,
              aftername     = ,
            }, 
            subsection = {
              name          = {,.},
              number        = {\arabic{subsection}},
              indent        = 2 em,

            }, 
            subsubsection = {
              name          = {(,)},
              number        = {\arabic{subsubsection}},
              aftername     = {\ },
              indent        = 2 em,
            }, 
            paragraph = {
              name          = {,)},
              number        = \arabic{paragraph},
              aftername     = ,
              indent        = 4 em,
            }
          }
      },
%    \end{macrocode}
%
% 设置外文章节标题。
%    \begin{macrocode}
    numbering / alpha   .code:n = 
      {
        \keys_set:nn { ctex }
          {
            chapter = {
              name          = {,.},
              number        = \Alph{chapter},
            },
            section = {
              name          = {(,)},
              number        = \Alph{section},
              aftername     = {\ },
            },
            subsection = {
              name          = {,.},
              number        = \alph{subsection},
              aftername     = {\ },
              numberformat  = \rmfamily,
            }, 
            subsubsection = {
              name          = {(,)},
              number        = \alph{subsubsection},
              aftername     = {\ },
              numberformat  = \rmfamily,
            }, 
            paragraph = {
              name          = {,)},
              number        = \alph{paragraph},
              aftername     = {\ },
              numberformat  = \rmfamily,
            },
          }
      },
%    \end{macrocode}
%
% 处理未知选项。
%    \begin{macrocode}
    numbering / unknown   .code:n = { \@@_error:n { unknown-value } }
  }
\@@_msg_new:nn { unknown-value }
  { The~ value~ "\l_keys_value_tl"~ is~ unknown. }
%    \end{macrocode}
%
% \begin{macro}{\@@_title:n}
% 手动生成章的标题,用于摘要、参考文献等。这些标题的字体大小
% 为小三,行距为 1.5 倍行距。
%    \begin{macrocode}
\cs_new_protected:Npn \@@_title:n #1
  {
    \group_begin:
      \ctexset 
        { 
          chapter / numbering = false,
          chapter / format    = \sffamily\bfseries\centering,
          chapter / titleformat = \large,
          chapter / afterskip = 3.5ex plus 0.5ex,
        }
      \chapter {#1}
    \group_end:
  }
\cs_generate_variant:Nn \@@_title:n { V }
%    \end{macrocode}
% \end{macro}
%
% \begin{macro}{\@@_abs_title:n}
% 手动生成中外文摘要页的标题。这些标题的字体大小
% 为小三,行距为单倍行距。
%    \begin{macrocode}
\cs_new_protected:Npn \@@_abs_title:n #1
  {
    \group_begin:
      \ctexset 
        { 
          chapter / numbering   = false,
          chapter / format      = \sffamily\bfseries\centering,
          chapter / titleformat = \large,
          chapter / beforeskip  = 2.25 bp,
          chapter / afterskip   = 2.25 bp,
        }
      \chapter* {#1}
    \group_end:
  }
\cs_generate_variant:Nn \@@_abs_title:n { V }
\cs_new_protected:Npn \@@_absEN_title:n #1
  {
    \group_begin:
      \ctexset
        {
          chapter / numbering   = false,
          chapter / format      = \rmfamily\bfseries\centering,
          chapter / titleformat = \large,
          chapter / beforeskip  = 2.25 bp,
          chapter / afterskip   = 2.25 bp,
        }
      \chapter* {#1}
    \group_end:
  }
\cs_generate_variant:Nn \@@_absEN_title:n { V }
%    \end{macrocode}
% \end{macro}
%
% \begin{macro}{\@@_toc_title:n}
% 手动生成目录页的标题。该标题的字体大小为小四,行距为单倍行距。
%    \begin{macrocode}
\cs_new_protected:Npn \@@_toc_title:n #1
  {
    \group_begin:
      \ctexset 
        { 
          chapter / numbering   = false,
          chapter / pagestyle   = empty,
          chapter / format      = \rmfamily\bfseries\centering,
          chapter / titleformat = \normalsize,
        }
      \chapter* {#1}
    \group_end:
  }
\cs_generate_variant:Nn \@@_toc_title:n { V }
%    \end{macrocode}
% \end{macro}
%
% \subsection{页眉页脚设置}
%
% 清除默认页眉页脚格式。
%    \begin{macrocode}
\fancyhf { }
%    \end{macrocode}
%
% 页眉样式。页眉居中表示,页眉内容为毕业论文(设计)的题目,
% 字体为宋体,字号为小五。
%    \begin{macrocode}
\fancyhead[C]   { \footnotesize \l_@@_info_title_tl }
%    \end{macrocode}
%
% 页脚样式。页脚居中表示,页脚内容为阿拉伯数字页码,字号为小五。
%    \begin{macrocode}
\fancyfoot[C]   { \footnotesize \thepage }
%    \end{macrocode}
%
%
% \begin{macro}{\cleardoublepage}
% 重定义 \tn{cleardoublepage},使得偶数页面在没有内容时也不显示页眉页脚,见
%    \begin{macrocode}
\RenewDocumentCommand \cleardoublepage { }
  {
    \clearpage
    \bool_if:NT \g_@@_twoside_bool
      {
        \int_if_odd:nF \c@page
          { \hbox:n { } \thispagestyle { empty } \newpage }
      }
  }
%    \end{macrocode}
% \end{macro}
%
% 由于 \pkg{ctex} 宏包使用 \texttt{heading} 选项后,会将页面格式设置为 \texttt{headings}。
% 因此需要在调用 \pkg{ctex} 后重新将 \cs{pagestyle} 设置为 \texttt{fancy}。
%    \begin{macrocode}
\pagestyle { fancy }
%    \end{macrocode}
%
% \subsection{脚注}
%
% \begin{macro}{\@makefntext}
% 重定义内部脚注文字命令,使生成的脚注无缩进。
%    \begin{macrocode}
\cs_set:Npn \@makefntext #1
  {
    \mode_leave_vertical:
    \textsuperscript{\@thefnmark} #1
  }
%    \end{macrocode}
% \end{macro}
%
% \subsection{图表绘制;浮动体}
%
% 设置浮动体 figure 和 table 的标题样式。
%    \begin{macrocode}
\captionsetup
  {
    font          = small,
    labelsep      = quad,
    justification = centering
  }
%    \end{macrocode}
%
% 设置英语标题的名称。
%    \begin{macrocode}
\captionsetup [ figure ] [ bi-second ] { name=Figure }
\captionsetup [ table  ] [ bi-second ] { name=Table  }
%    \end{macrocode}
%
% 
% \begin{macro}{\@makefntext}
% 重定义图表编号。
%    \begin{macrocode}
\cs_set:Npn \thefigure
  {  \thechapter - \int_to_arabic:n { \c@figure } }
\cs_set:Npn \thetable
  {  \thechapter - \int_to_arabic:n { \c@table  } }
%    \end{macrocode}
% \end{macro}
%
% \subsection{封面}
%
% \subsubsection{信息录入}
%
% \begin{variable}{\l_@@_info_title_tl,
%   \l_@@_info_titleEN_tl,
%   \l_@@_info_author_tl,
%   \l_@@_info_studentID_tl,
%   \l_@@_info_department_tl,
%   \l_@@_info_major_tl,
%   \l_@@_info_supervisor_tl,
%   \l_@@_info_academicTitle_tl,
%   \l_@@_info_keywords_clist,
%   \l_@@_info_keywordsEN_clist}
% 封面所需的一些字段(中外文标题、作者姓名、学号、院系、专业、
% 导师姓名、导师职称、中外文关键词)。
%    \begin{macrocode}
\clist_map_inline:nn
  {
    title, titleEN, author, studentID, department, major, supervisor
    academicTitle
  }
  { \tl_new:c { l_@@_info_ #1 _tl } }
\clist_new:N \l_@@_info_keywords_clist
\clist_new:N \l_@@_info_keywordsEN_clist
%    \end{macrocode}
% \end{variable}
%
% \begin{variable}{\l_@@_info_year_int,
%   \l_@@_info_month_int,
%   \l_@@_info_graduation_year_int}
% 论文完成年份、月份与毕业年份。
%    \begin{macrocode}
\int_new:N   \l_@@_info_year_int
\int_new:N   \l_@@_info_month_int
\int_new:N   \l_@@_info_graduation_year_int
%    \end{macrocode}
% \end{variable}
%
%
% 定义 \texttt{ecnu/info} 键值类。
%    \begin{macrocode}
\keys_define:nn { ecnu / info }
  {
%    \end{macrocode}
%
% \begin{macro}{info/title,info/titleEN}
% 论文中外文标题。
%    \begin{macrocode}
    title           .tl_set:N     = \l_@@_info_title_tl,
    titleEN         .tl_set:N     = \l_@@_info_titleEN_tl,
%    \end{macrocode}
% \end{macro}
%
% \begin{macro}{info/author}
% 论文作者姓名。
%    \begin{macrocode}
    author          .tl_set:N     = \l_@@_info_author_tl,
%    \end{macrocode}
% \end{macro}
%
% \begin{macro}{info/studentID}
% 论文作者学号。
%    \begin{macrocode}
    studentID       .tl_set:N     = \l_@@_info_studentID_tl,
%    \end{macrocode}
% \end{macro}
%
% \begin{macro}{info/department}
% 论文作者所在院系。
%    \begin{macrocode}
    department      .tl_set:N     = \l_@@_info_department_tl,
%    \end{macrocode}
% \end{macro}
%
% \begin{macro}{info/major}
% 论文作者就读专业。
%    \begin{macrocode}
    major           .tl_set:N     = \l_@@_info_major_tl,
%    \end{macrocode}
% \end{macro}
%
% \begin{macro}{info/supervisor}
% 论文指导教师姓名。
%    \begin{macrocode}
    supervisor      .tl_set:N     = \l_@@_info_supervisor_tl,
%    \end{macrocode}
% \end{macro}
%
% \begin{macro}{info/academicTitle}
% 论文指导教师职称。
%    \begin{macrocode}
    academicTitle   .tl_set:N     = \l_@@_info_academicTitle_tl,
%    \end{macrocode}
% \end{macro}
%
% \begin{macro}{info/keywords,info/keywordsEN}
% 论文中外文关键词。
%    \begin{macrocode}
    keywords        .clist_set:N  = \l_@@_info_keywords_clist,
    keywordsEN      .clist_set:N  = \l_@@_info_keywordsEN_clist,
%    \end{macrocode}
% \end{macro}
%
% \begin{macro}{info/year, info/month}
% 论文完成年份与月份。
%    \begin{macrocode}
    year            .int_set:N    = \l_@@_info_year_int,
    month           .int_set:N    = \l_@@_info_month_int,
%    \end{macrocode}
% \end{macro}
%
% \begin{macro}{info/graduationYear}
% 论文作者毕业年份。
%    \begin{macrocode}
    graduationYear  .int_set:N    = \l_@@_info_graduation_year_int,
  }
%    \end{macrocode}
% \end{macro}
%
% \begin{macro}{\theyear}
% 记录论文完成年份。默认为当前年份。
%    \begin{macrocode}
\NewDocumentCommand \theyear { } 
  {
    \int_compare:nNnTF { \l__ecnu_info_year_int } = { \c_zero_int } 
      { \int_set_eq:NN \l__ecnu_info_year_int \c_sys_year_int } 
      { }
    \int_use:N \l__ecnu_info_year_int
  }
%    \end{macrocode}
% \end{macro}
%
% \begin{macro}{\themonth}
% 记录论文完成月份。默认为当前月份。
%    \begin{macrocode}
\NewDocumentCommand \themonth { } 
  {
    \int_compare:nNnTF { \l__ecnu_info_month_int } = { \c_zero_int } 
      { \int_set_eq:NN \l__ecnu_info_month_int \c_sys_month_int }
      { }
    \int_use:N \l__ecnu_info_month_int
  }
%    \end{macrocode}
% \end{macro}
%
% \begin{macro}{\graduationYear}
% 记录论文作者毕业届数。若 \texttt{info/graduationYear} 未设置,则使用
% \texttt{info/year} 的值;若\texttt{info/year} 的值亦未设置,则使用
% 当前年份。
%    \begin{macrocode}
\NewDocumentCommand \graduationYear { } 
  {
    \int_compare:nNnTF { \l__ecnu_info_graduation_year_int } = { \c_zero_int } 
      {
        \int_compare:nNnTF { \l__ecnu_info_year_int } = { \c_zero_int } 
          {
            \int_set_eq:NN  \l__ecnu_info_graduation_year_int 
                            \c_sys_year_int
          } 
          {
            \int_set_eq:NN  \l__ecnu_info_graduation_year_int 
                            \l__ecnu_info_year_int
          }
      }
      { }
    \int_use:N  \l__ecnu_info_graduation_year_int
  }
%    \end{macrocode}
% \end{macro}
%
% \subsubsection{定义内部函数}
%
% \begin{macro}{\@@_spread_box:nn,\@@_spread_box:no}
% 分散对齐的水平盒子。
% \begin{arguments}
%   \item 宽度
%   \item 内容
% \end{arguments}
% 利用 \cs{tl_map_inline:nn} 在字符间插入 \texttt{hfil};紧随其后的 \tn{unskip}
% 将会去掉最后一个 \texttt{hfil}。
%    \begin{macrocode}
\cs_new_protected:Npn \@@_spread_box:nn #1#2
  {
    \mode_leave_vertical:
    \hbox_to_wd:nn {#1} { \tl_map_inline:nn {#2} { ##1 \hfil } \unskip }
  }
\cs_generate_variant:Nn \@@_spread_box:nn { no }
%    \end{macrocode}
% \end{macro}
%
% \begin{macro}{\@@_center_box:nn,\@@_center_box:Vn}
% 居中对齐的水平盒子。
%    \begin{macrocode}
\cs_new_protected:Npn \@@_center_box:nn #1#2
  {
    \mode_leave_vertical:
    \hbox_to_wd:nn {#1} { \hfil #2 \hfil }
  }
\cs_generate_variant:Nn \@@_center_box:nn { Vn }
%    \end{macrocode}
% \end{macro}
%
% \begin{macro}{\@@_fixed_width_box:nn}
% 限宽盒子(允许换行)。
%    \begin{macrocode}
\cs_new:Npn \@@_fixed_width_box:nn #1#2
  { \parbox {#1} {#2} }
%    \end{macrocode}
% \end{macro}
%
% \begin{macro}{\@@_fixed_width_center_box:nn}
% 居中对齐的限宽盒子(允许换行)。
%    \begin{macrocode}
\cs_new:Npn \@@_fixed_width_center_box:nn #1#2
  { \parbox {#1} { \centering #2 } }
%    \end{macrocode}
% \end{macro}
%
% \begin{macro}{\@@_get_text_width:Nn,\@@_get_text_width:NV}
% 获取文本宽度,并存入 \texttt{dim} 型变量。
% \begin{arguments}
%   \item \texttt{dim} 型变量
%   \item 内容
% \end{arguments}
%    \begin{macrocode}
\cs_new:Npn \@@_get_text_width:Nn #1#2
  {
    \hbox_set:Nn \l_@@_tmpa_box {#2}
    \dim_set:Nn #1 { \box_wd:N \l_@@_tmpa_box }
  }
\cs_generate_variant:Nn \@@_get_text_width:Nn { NV }
%    \end{macrocode}
% \end{macro}
%
% \begin{macro}{\@@_line_spread:N,\@@_line_spread:n}
% 设置行距。
%    \begin{macrocode}
\cs_new:Npn \@@_line_spread:N #1
  { \linespread { \fp_use:N #1 } \selectfont }
\cs_new:Npn \@@_line_spread:n #1
  { \linespread {#1} \selectfont }
%    \end{macrocode}
% \end{macro}
%
% \subsubsection{预设文字}
% 
% 模板中的一些预设文字,用于封面、声明页等部分。
%    \begin{macrocode}
\tl_const:Nn \c_@@_info_name_tl          { 姓名      }
\tl_const:Nn \c_@@_info_studentID_tl     { 学号      }
\tl_const:Nn \c_@@_info_department_tl    { 学院      }
\tl_const:Nn \c_@@_info_major_tl         { 专业      }
\tl_const:Nn \c_@@_info_supervisor_tl    { 指导教师   }
\tl_const:Nn \c_@@_info_academicTitle_tl { 职称      }
\tl_const:Nn \c_@@_schoolNum_tl          { 学校代码   }
\tl_const:Nn \c_@@_year_tl               { 年 }
\tl_const:Nn \c_@@_month_tl              { 月 }
\tl_const:Nn \c_@@_day_tl                { 日 }
\tl_const:Nn \c_@@_thesisInfo_tl         { 届本科生学士学位论文 }
\tl_const:Nn \c_@@_date_tl               { 日期 }
\tl_const:Nn \c_@@_orig_sign_tl          { 承诺人签名 }
\tl_const:Nn \c_@@_author_sign_tl        { 作者签名 }
\tl_const:Nn \c_@@_supervisor_sign_tl    { 导师签名 }
\tl_const:Nn \c_@@_name_orig_tl  { 华东师范大学学位论文诚信承诺 }
\tl_const:Nn \c_@@_name_auth_tl  { 华东师范大学学位论文使用授权说明 }
\tl_const:Nn \c_@@_orig_text_tl
  {
    \par
    本毕业论文是本人在导师指导下独立完成的,内容真实、可靠。本人在撰写
    毕业论文过程中不存在请人代写、抄袭或者剽窃他人作品、伪造或者篡改数据
    以及其他学位论文作假行为。\par
    本人清楚知道学位论文作假行为将会导致行为人受到不授予/撤销学位、开除
    学籍等处理(处分)决定。本人如果被查证在撰写本毕业论文过程中存在学位
    论文作假行为,愿意接受学校依法作出的处理(处分)决定。
  }
\tl_const:Nn \c_@@_auth_text_tl
  {
    \par
    本论文的研究成果归华东师范大学所有,本论文的研究内容不得以其它单位的
    名义发表。本学位论文作者和指导教师完全了解华东师范大学有关保留、使用
    学位论文的规定,即:学校有权保留并向国家有关部门或机构送交论文的复印
    件和电子版,允许论文被查阅和借阅;本人授权华东师范大学可以将论文的全部
    或部分内容编入有关数据库进行检索、交流,可以采用影印、缩印或其他复制
    手段保存论文和汇编本学位论文。\par
    保密的毕业论文(设计)在解密后应遵守此规定。
  }
\tl_const:Nn \c_@@_name_toc_tl          { 目录 }
\tl_const:Nn \c_@@_name_abstract_tl     { 摘要 }
\tl_const:Nn \c_@@_name_abstractEN_tl   { Abstract }
\tl_const:Nn \c_@@_name_keywords_tl     { 关键词 }
\tl_const:Nn \c_@@_name_keywordsEN_tl   { Keywords }
\tl_const:Nn \c_@@_colon_tl             { : }
\tl_const:Nn \c_@@_name_appendix_tl  { 附录 }
\tl_const:Nn \c_@@_name_acknowledgement_tl  { 致谢 }
%    \end{macrocode}
%
% \subsubsection{封面各部件}
%
% \begin{macro}{\@@_cover_id:}
% 封面首行信息部件,包括毕业届别以及学校代码。
%    \begin{macrocode}
\cs_new_protected:Npn \@@_cover_id:
  {
    \@@_fixed_width_box:nn { \textwidth }
      {
        \bfseries\zihao { 4 }
        \graduationYear \c_@@_thesisInfo_tl \hfill
        \c_@@_schoolNum_tl \c_@@_colon_tl \underline{10269}
      }
  }
%    \end{macrocode}
% \end{macro}
%
% \begin{macro}{\@@_orig_sign_area:}
% 原创性声明。
%    \begin{macrocode}
\cs_new_protected:Npn \@@_orig_sign_area:
  {
    \c_@@_orig_sign_tl \c_@@_colon_tl
    \hbox_to_wd:nn { 15 em } { }
    \c_@@_date_tl \c_@@_colon_tl
    \hbox_to_wd:nn { 3.4 em } { } \c_@@_year_tl
    \hbox_to_wd:nn { 1.5 em } { } \c_@@_month_tl
    \hbox_to_wd:nn { 1.5 em } { } \c_@@_day_tl
  }
%    \end{macrocode}
% \end{macro}
%
% \begin{macro}{\@@_auth_sign_area:}
% 论文使用授权声明。
%    \begin{macrocode}
\cs_new_protected:Npn \@@_auth_sign_area:
  {
    \c_@@_author_sign_tl \c_@@_colon_tl
    \hbox_to_wd:nn { 5.5 em } { }
    \c_@@_supervisor_sign_tl \c_@@_colon_tl
    \hbox_to_wd:nn { 5.5 em } { }
    \c_@@_date_tl \c_@@_colon_tl
    \hbox_to_wd:nn { 3.4 em } { } \c_@@_year_tl
    \hbox_to_wd:nn { 1.5 em } { } \c_@@_month_tl
    \hbox_to_wd:nn { 1.5 em } { } \c_@@_day_tl
  }
%    \end{macrocode}
% \end{macro}
%
%
% \begin{macro}{\@@_cover_info:}
% 论文封面中的个人信息(姓名、学号、院系、专业、导师及其职称)和论文完成年月。
%    \begin{macrocode}
\cs_new_protected:Npn \@@_cover_info:
  {
    \begin{minipage} [ c ] { \textwidth }
      \centering \zihao { 4 }
      \clist_set:Nx \l_@@_tmpa_clist
        {
          \c_@@_info_name_tl,
          \c_@@_info_studentID_tl,
          \c_@@_info_department_tl,
          \c_@@_info_major_tl,
          \c_@@_info_supervisor_tl,
          \c_@@_info_academicTitle_tl
        }
      \clist_set:Nx \l_@@_tmpb_clist
        {
          { \l_@@_info_author_tl        },
          { \l_@@_info_studentID_tl     },
          { \l_@@_info_department_tl    },
          { \l_@@_info_major_tl         },
          { \l_@@_info_supervisor_tl    },
          { \l_@@_info_academicTitle_tl }
        }
      \bool_until_do:nn
        { \clist_if_empty_p:N \l_@@_tmpa_clist }
        {
          \bfseries
          \clist_pop:NN \l_@@_tmpa_clist \l_@@_tmpa_tl
          \clist_pop:NN \l_@@_tmpb_clist \l_@@_tmpb_tl
          \@@_spread_box:no { 4 em } { \l_@@_tmpa_tl }
          \c_@@_colon_tl
          \underline{\@@_center_box:nn { 6 cm } { \l_@@_tmpb_tl }}
          \skip_vertical:n { 1 ex }
        }
      \skip_vertical:n { 1 em }
      \theyear  \c_@@_year_tl
      \themonth \c_@@_month_tl
      \skip_vertical:n { 1 ex }
    \end{minipage}
  }
%    \end{macrocode}
% \end{macro}
%
% \begin{macro}{style/logoResource}
% 定义封面 logo 接口。
%    \begin{macrocode}
\tl_new:N \l_@@_logo_resource_tl
\keys_define:nn { ecnu / style }
  {
    logoResource .tl_set:N = \l_@@_logo_resource_tl
  }
%    \end{macrocode}
% \end{macro}
%
% \begin{macro}{\@@_cover_logo:}
% 封面 logo 部件
%    \begin{macrocode}
\cs_new_protected:Npn \@@_cover_logo:
  { 
    \tl_if_empty:NF \l_@@_logo_resource_tl
      {
        \begin{minipage} [ c ] { \textwidth }
          \centering
          \includegraphics{\l_@@_logo_resource_tl}
        \end{minipage}
      }
  }
%    \end{macrocode}
% \end{macro}
%
% \subsubsection{封面实现接口}
%
%    \begin{macrocode}
\tl_new:N \l_@@_cover_template_tl
\DeclareObjectType { ecnu / cover } { \c_zero_int }
\cs_new_protected:Npn \@@_cover_declare_template_interface:nn #1#2
  { \DeclareTemplateInterface { ecnu / cover } {#1} { \c_zero_int } {#2} }
\cs_new_protected:Npn \@@_cover_declare_template_code:nnn #1#2#3
  { \DeclareTemplateCode { ecnu / cover } {#1} { \c_zero_int } {#2} {#3} }
\cs_generate_variant:Nn \@@_cover_declare_template_interface:nn { nx  }
\cs_generate_variant:Nn \@@_cover_declare_template_code:nnn     { nxn }
\cs_new:Npn \@@_cover_key_type:n #1
  {
    #1 / content     : tokenlist,
    #1 / format      : tokenlist,
    #1 / bottom-skip : skip,
    #1 / align       : choice { left, right, center, normal } = normal,
  }
\cs_new:Npn \@@_cover_key_binding:n #1
  {
    #1 / content     =
      \exp_not:c
        { l_@@_cover / \l_@@_cover_template_tl / #1 / content_tl  },
    #1 / format      =
      \exp_not:c
        { l_@@_cover / \l_@@_cover_template_tl / #1 / format_tl   },
    #1 / bottom-skip =
      \exp_not:c
        { l_@@_cover / \l_@@_cover_template_tl / #1 / bottom_skip },
    #1 / align       =
      {
        left   =
          \exp_not:N \cs_set_protected:cpn
            { @@_cover / \l_@@_cover_template_tl / #1 / align:n }
            \exp_not:n {##1}
            {
              \exp_not:n
                {
                  \group_begin:
                    \flushleft ##1 \endflushleft
                  \group_end:
                }
            },
        right  =
          \exp_not:N \cs_set_protected:cpn
            { @@_cover / \l_@@_cover_template_tl / #1 / align:n }
            \exp_not:n {##1}
            {
              \exp_not:n
                {
                  \group_begin:
                    \flushright ##1 \endflushright
                  \group_end:
                }
            },
        center =
          \exp_not:N \cs_set_protected:cpn
            { @@_cover / \l_@@_cover_template_tl / #1 / align:n }
            \exp_not:n {##1}
            {
              \exp_not:n
                {
                  \group_begin:
                    \center ##1 \endcenter
                  \group_end:
                }
            },
        normal =
          \exp_not:N \cs_set_protected:cpn
            { @@_cover / \l_@@_cover_template_tl / #1 / align:n }
            \exp_not:n {##1}
            { \exp_not:n { \group_begin: ##1 \group_end: } }
      },
  }
\cs_new_protected:Npn \@@_cover_declare_template:nn #1#2
  {
    \tl_set:Nn \l_@@_cover_template_tl {#1}
    \@@_cover_declare_template_interface:nx {#1}
      {
        format      : tokenlist,
        top-skip    : skip,
        bottom-skip : skip,
        \clist_map_function:nN {#2} \@@_cover_key_type:n
      }
    \tl_new:c   { l_@@_cover / #1 / format_tl   }
    \skip_new:c { l_@@_cover / #1 / top_skip    }
    \skip_new:c { l_@@_cover / #1 / bottom_skip }
    \clist_map_inline:nn {#2}
      {
        \tl_new:c   { l_@@_cover / #1 / ##1 / content_tl  }
        \tl_new:c   { l_@@_cover / #1 / ##1 / format_tl   }
        \skip_new:c { l_@@_cover / #1 / ##1 / bottom_skip }
      }
    \@@_cover_declare_template_code:nxn {#1}
      {
        format      = \exp_not:c { l_@@_cover / #1 / format_tl   },
        top-skip    = \use:c     { l_@@_cover / #1 / top_skip    },
        bottom-skip = \use:c     { l_@@_cover / #1 / bottom_skip },
        \clist_map_function:nN {#2} \@@_cover_key_binding:n
      }
      {
        \AssignTemplateKeys
        \tl_use:c        { l_@@_cover / #1 / format_tl }
        \clist_map_inline:nn {#2}
          { 
            \noindent
            \use:c { @@_cover / #1 / ####1 / align:n }
              {
                \tl_use:c { l_@@_cover / #1 / ####1 / format_tl  }
                \tl_use:c { l_@@_cover / #1 / ####1 / content_tl }
                \par
              }
            \@@_vspace:c { l_@@_cover / #1 / ####1 / bottom_skip }
          }
        \@@_vspace:c { l_@@_cover / #1 / bottom_skip }
        \clearpage{\thispagestyle{empty}\cleardoublepage}
      }
  }
\NewDocumentCommand \DeclareCoverTemplate { m m }
  { \@@_cover_declare_template:nn {#1} {#2} }
\DeclareCoverTemplate { cover-i   }
  { id, logo, title, titleEN, info }
\DeclareCoverTemplate { cover-ii  }
  {
    title-space,
    originality-name,
    originality-text,
    originality-sign,
    authorization-name,
    authorization-text,
    authorization-sign
  }
\renewcommand{\ULthickness}{1pt}
\DeclareInstance { ecnu / cover } { cover-i-default } { cover-i }
  {
    bottom-skip            = 0 pt plus 0.5 fill,
    id       / content     = \@@_cover_id:,
    logo     / content     = \@@_cover_logo:,
    title    / content     =
      \@@_fixed_width_center_box:nn
        { 0.9 \textwidth } 
        { \expandafter\uline\expandafter{\l_@@_info_title_tl} },
    titleEN  / content     = 
      \@@_fixed_width_center_box:nn
        { 0.9 \textwidth } 
        { \expandafter\uline\expandafter{\l_@@_info_titleEN_tl} },
    info     / content     = \@@_cover_info:,
    title    / format      = \@@_line_spread:n { 1.1 } \zihao { 1 } \bfseries,
    titleEN  / format      = \@@_line_spread:n { 1.1 } \zihao { 1 } \bfseries,
    id       / bottom-skip = 0 pt plus 0.5 fill,
    logo     / bottom-skip = 0 pt plus 0.3 fill,
    titleEN  / bottom-skip = 0 pt plus 2.5 fill,
    id       / align       = right,
    logo     / align       = center,
    title    / align       = center,
    titleEN  / align       = center,
    info     / align       = center,
  }
\DeclareInstance { ecnu / cover } { cover-ii-default } { cover-ii }
  {
    format                            = \@@_line_spread:n { 1.625 },
    top-skip                          = 0 pt plus 0.2 fill,
    bottom-skip                       = 0 pt plus 2.5 fill,
    originality-name    / content     = \c_@@_name_orig_tl,
    originality-text    / content     = \c_@@_orig_text_tl,
    originality-sign    / content     = \@@_orig_sign_area:,
    authorization-name  / content     = \c_@@_name_auth_tl,
    authorization-text  / content     = \c_@@_auth_text_tl,
    authorization-sign  / content     = \@@_auth_sign_area:,
    originality-name    / format      =
      \@@_line_spread:n { 1.2 } \zihao { 4 } \bfseries,
    authorization-name  / format      =
      \@@_line_spread:n { 1.2 } \zihao { 4 } \bfseries,
    title-space         / bottom-skip = 1 em plus 0.2 fill,
    originality-text    / bottom-skip = -0.5 em,
    originality-sign    / bottom-skip = 0 pt plus 0.75 fill,
    authorization-name  / bottom-skip = 0 pt,
    authorization-text  / bottom-skip = 1 em,
    originality-name    / align       = center,
    originality-sign    / align       = right,
    authorization-name  / align       = center,
    authorization-sign  / align       = right,
  }
\tl_new:N \l_@@_declaration_page_tl
\keys_define:nn { ecnu / style }
  {
    declarePageResource .code:n =
      {
        \tl_set_eq:NN \l_@@_declaration_page_tl \l_keys_value_tl
        \RequirePackage { pdfpages }
      },
  }
\NewDocumentCommand \makecoveri { }
  {
    \thispagestyle { empty }
    \UseInstance { ecnu / cover } { cover-i-default }
  }
\NewDocumentCommand \makecoverii { }
  {
    \cleardoublepage
    \thispagestyle { empty }
    \tl_set:Nn \thepage { C }
    \bool_if:NTF \g_@@_decl_page_bool
      {
        \tl_if_empty:NTF \l__ecnu_declaration_page_tl
          { \UseInstance { ecnu / cover } { cover-ii-default } }
          { \includepdf { \l__ecnu_declaration_page_tl } }
      }
      {}
  }
%    \end{macrocode}
%
%
%
%
%
%
%
%
%
%
%
%
%
%
%
%
%
%
%
%
%
%
%
%
%
%
%
%
%
%
%
%
%
%
%
%
%
%
% \subsection{目录}
%
% 设置目录标题以及目录层级。目录层级仅显示一级和二级标题。
%    \begin{macrocode}
\keys_set:nn { ctex }
  {
    contentsname  = \c_@@_name_toc_tl,
    tocdepth      = 1,
  }
%    \end{macrocode}
%
% 使用 \pkg{tocloft} 设置目录格式。
%    \begin{macrocode}
\PassOptionsToPackage { titles } { tocloft }
\RequirePackage { tocloft }
%    \end{macrocode}
%
% \begin{macro}{\cftchapleader}
% 设置章节标题与对应页码之间的引导点格式。
%    \begin{macrocode}
\RenewDocumentCommand \cftchapleader { }
  {\bfseries\cftdotfill{1.0}}
%    \end{macrocode}
% \end{macro}
%
% \begin{macro}{\cftsecleader}
% 设置小节标题与对应页码之间的引导点格式。
%    \begin{macrocode}
\RenewDocumentCommand \cftsecleader { }
  {\cftdotfill{1.0}}
%    \end{macrocode}
% \end{macro}
%
% 设置章节标题与小节标题相关的间距。
%    \begin{macrocode}
\setlength{\cftbeforechapskip}{6pt}
\setlength{\cftchapnumwidth}{0pt}
\setlength{\cftsecnumwidth}{0pt}
\RenewDocumentCommand \@tocrmarg { } { 0pt }
%    \end{macrocode}
%
% \begin{macro}{\cftchapfillnum}
% 去除目录中章节标题所对应的页码前的空白。
%    \begin{macrocode}
\RenewDocumentCommand \cftchapfillnum { m } 
  {
    {\cftchapleader}\nobreak
    {\cftchappagefont #1}\cftchapafterpnum\par
  }
%    \end{macrocode}
% \end{macro}
%
% \begin{macro}{\cftsecfillnum}
% 去除目录中小节标题所对应的页码前的空白。
%    \begin{macrocode}
\RenewDocumentCommand \cftsecfillnum { m }
  {
    {\cftsecleader}\nobreak
    {\cftsecpagefont #1}\cftsecafterpnum\par
  }
%    \end{macrocode}
% \end{macro}
%
% \begin{macro}{\tableofcontents}
% 修改 \cs{tableofcontents}。来自于 \LaTeXe{} 标准文档类
% \file{book.cls}
%    \begin{macrocode}
\RenewDocumentCommand \tableofcontents { }
  {
    \pagestyle{ empty }
    \@@_toc_title:V \contentsname
    \group_begin:
      \@@_line_spread:n {1.2}
      \@starttoc{toc}
    \group_end:
    \cleardoublepage
    \pagestyle{ fancy }
  }
%    \end{macrocode}
% \end{macro}
%
% 在目录中,若某章节下有一个或多个小节,则目录中该章节与该小节之间应有 6 pt 的
% 竖直间距。
%    \begin{macrocode}
\@@_preto_cmd:Nn \section
  {
    \int_compare:nNnTF { \c@section } = { \c_zero_int }
      { \addtocontents{toc}{\protect\vspace{6pt}} }
      { }
  }
%    \end{macrocode}
%
% \subsection{摘要}
%
% \begin{environment}{abstract}
% \begin{environment}{abstractEN}
% 定义中外文摘要环境。
%    \begin{macrocode}
\RenewDocumentEnvironment { abstract  } { }
  { \@@_abstract_begin:    } { \@@_abstract_end:      }
\NewDocumentEnvironment { abstractEN } { }
  { \@@_abstractEN_begin: } { \@@_abstractEN_end:   }
%    \end{macrocode}
% \end{environment}
% \end{environment}
%
% \begin{macro}{\@@_abstract_begin:,\@@_abstractEN_begin:}
% 中外文摘要页环境起始部分。在环境起始部分手动添加中外文标题,并将该
% “摘要”(“Abstract”)添加至目录;将摘要部分字体大小设置为五号。
%    \begin{macrocode}
\cs_new_protected:Npn \@@_abstract_begin:
  {
    \addcontentsline{toc}{chapter}{\c_@@_name_abstract_tl}
    \@@_abs_title:V \l_@@_info_title_tl
    \small
    \hbox:n {} \mode_leave_vertical: \par \noindent
    \group_begin:
      \rmfamily\bfseries \c_@@_name_abstract_tl \c_@@_colon_tl
    \group_end:
    \par
  }
\cs_new_protected:Npn \@@_abstractEN_begin:
  {
    \addcontentsline{toc}{chapter}{ABSTRACT}
    \@@_absEN_title:V \l_@@_info_titleEN_tl
    \small
    \hbox:n {} \mode_leave_vertical: \par \noindent
    \group_begin:
      \rmfamily\bfseries \c_@@_name_abstractEN_tl \c_@@_colon_tl
    \group_end:
    \par
  }
%    \end{macrocode}
% \end{macro}
%
% \begin{macro}{\@@_abstract_end:,\@@_abstractEN_end:}
% 摘要页标题环境结束部分。在摘要部分正文末尾附上关键词。
%    \begin{macrocode}
\cs_new_protected:Npn \@@_abstract_end:
  {
    \@@_keywords:nNn
      { \rmfamily\bfseries \c_@@_name_keywords_tl \c_@@_colon_tl }
      \l_@@_info_keywords_clist { , }
    \cleardoublepage
  }
\cs_new_protected:Npn \@@_abstractEN_end:
  {
    \@@_keywords:nNn
      { \rmfamily\bfseries \c_@@_name_keywordsEN_tl \c_@@_colon_tl }
      \l_@@_info_keywordsEN_clist { ,~ }
    \cleardoublepage
  }
%    \end{macrocode}
% \end{macro}
%
% \begin{macro}{\@@_keywords:nNn}
% 中外文摘要页关键词。
%    \begin{macrocode}
\cs_new_protected:Npn \@@_keywords:nNn #1#2#3
  {
    \par \mode_leave_vertical: \par \noindent
    \@@_get_text_width:Nn \l_@@_tmpa_dim {#1}
    \group_begin: #1 \group_end:
    \parbox [t] { \dim_eval:n { \textwidth - \l_@@_tmpa_dim } }
      {
        \clist_use:Nn #2 {#3} \par
        \cs_gset:Npx \@@_keywords_prevdepth:
          { \dim_use:N \tex_prevdepth:D }
      }
  }
%    \end{macrocode}
% \end{macro}
%
%
% \subsection{附录}
%
% \begin{environment}{appendix}
% 定义附录环境。
%    \begin{macrocode}
\RenewDocumentEnvironment { appendix } { }
  { \@@_appendix_begin: } { \@@_appendix_end: }
%    \end{macrocode}
% \end{environment}
%
% \begin{macro}{\@@_appendix_begin:}
% 添加附录标题,将附录环境内正文的字号调整为五号。
%    \begin{macrocode}
\cs_new_protected:Npn \@@_appendix_begin:
  { 
    \cleardoublepage
    \@@_title:V \c_@@_name_appendix_tl
    \setcounter{chapter}{0}
    \setcounter{section}{0}
    \small
  }
%    \end{macrocode}
% \end{macro}
%
% \begin{macro}{\@@_appendix_end:}
% 在打印模式下,保证附录部分为偶数页。
%    \begin{macrocode}
\cs_new_protected:Npn \@@_appendix_end:
  {
    \cleardoublepage
  }
%    \end{macrocode}
% \end{macro}
%
%
% \subsection{致谢}
%
% \begin{environment}{acknowledgement}
% 致谢环境。
%    \begin{macrocode}
\NewDocumentEnvironment { acknowledgement } { }
  { \@@_acknowledgement_begin: }
  { \@@_acknowledgement_end: }
%    \end{macrocode}
% \end{environment}
%
% \begin{macro}{\@@_acknowledgements_begin:}
% 致谢页标题。将致谢页环境内正文的字号调整为五号。
%    \begin{macrocode}
\cs_new_protected:Npn \@@_acknowledgement_begin:
  {
    \cleardoublepage
    \@@_title:V \c_@@_name_acknowledgement_tl
    \small
  }
%    \end{macrocode}
% \end{macro}
%
% \begin{macro}{\@@_acknowledgement_begin:}
% 在打印模式下,保证致谢页为偶数页。
%    \begin{macrocode}
\cs_new_protected:Npn \@@_acknowledgement_end:
  {
    \cleardoublepage
  }
%    \end{macrocode}
% \end{macro}
%
% \subsection{参考文献}
%
% 模版使用 \cs{biblatex} 和 Biber 处理参考文献。
%
% \begin{variable}{\l_@@_bib_resource_tl}
% 声明变量。
%    \begin{macrocode}
\tl_new:N \l_@@_bib_resource_tl
%    \end{macrocode}
% \end{variable}
%
% \begin{macro}{style/fontCJK}
% 定义参考文献数据库路径。
%    \begin{macrocode}
\keys_define:nn { ecnu / style }
  {
    bibResource .tl_set:N = \l_@@_bib_resource_tl
  }
%    \end{macrocode}
% \end{macro}
%
% \pkg{biblatex} 会写入 begindocument/before 钩子。因此需在其之前通过
% \texttt{env/document/begin} 钩子载入 \pkg{biblatex} 宏包。
%    \begin{macrocode}
\@@_gadd_ltxhook:nn { env/document/begin }
{
  \@@_biblatex_pre_setup:
  \RequirePackage { biblatex }
  \@@_biblatex_post_setup:
}
%    \end{macrocode}
%
% \begin{macro}{\@@_biblatex_pre_setup:, \@@_biblatex_post_setup:}
% biblatex 宏包相关设置。参考文献处理后端为 biber,样式遵循
% GB/T 7714-2015 标准。
%    \begin{macrocode}
\cs_new_protected:Npn \@@_biblatex_pre_setup:
  {
    \PassOptionsToPackage 
    {
      backend   = biber,
      style     = gb7714-2015,
      seconds   = true,
      hyperref  = true,
      gbpub     = false,
      gbpunctin = false,
    } 
    { biblatex }
  }
%    \end{macrocode}
% 
% 设置参考文献相关格式。参考文献字体为五号,参考文献间
% 没有额外间距。
%    \begin{macrocode}
\cs_new_protected:Npn \@@_biblatex_post_setup:
  {
    \exp_args:NV \addbibresource \l_@@_bib_resource_tl
    \@@_biblatex_allow_url_break:
    \RenewDocumentCommand \bibfont { } { \small }
    \setlength{\bibitemsep}{0pt}
    \defbibheading { bibliography } [ \bibname ] 
      { \@@_title:n {##1} }
  }
%    \end{macrocode}
% \end{macro}
%
% \begin{macro}{\@@_biblatex_allow_url_break}
% \pkg{biblatex} 下允许 URL 在字母、数字和一些特殊符号处断行。
%    \begin{macrocode}
\cs_new:Npn \@@_biblatex_allow_url_break:
  {
    \int_set_eq:NN \c@biburlucpenalty  \c_one_int
    \int_set_eq:NN \c@biburlnumpenalty \c_one_int
    \int_set_eq:NN \c@biburllcpenalty  \c_one_int
  }
%    \end{macrocode}
% \end{macro}
%
% \begin{macro}{\PrintReference}
% 用于生成参考文献的命令。
%    \begin{macrocode}
\NewDocumentCommand \PrintReference {} 
  {
    \printbibliography
    \cleardoublepage
  }
%    \end{macrocode}
% \end{macro}
%
% 在导言区末尾引入 \pkg{hyperref} 宏包,以尽可能减少与其他宏包
% 间的冲突。
%    \begin{macrocode}
\PassOptionsToPackage { hidelinks, hypertexnames=false } { hyperref }
\RequirePackage { hyperref }
%    \end{macrocode}
%
% \subsection{前导页}
%
% 设置前导页。前导页的页边距与正文有不同。
%    \begin{macrocode}
\AtBeginDocument
  {
    \newgeometry
      {
        top     = 2.0 cm, 
        bottom  = 2.0 cm,
        left    = 3.0 cm, 
        right   = 3.0 cm
      }
    \begin{titlepage}
      \makecoveri \newpage \makecoverii
    \end{titlepage}
    \restoregeometry
  }
%    \end{macrocode}
%
% \subsection{用户接口}
%
% \begin{macro}{info,style}
% 定义元(meta)键值对。
%    \begin{macrocode}
\keys_define:nn { ecnu }
  {
    info  .meta:nn = { ecnu / info  } {#1},
    style .meta:nn = { ecnu / style } {#1}
  }
%    \end{macrocode}
% \end{macro}
%
% \begin{macro}{\ecnuSetup}
% 用户设置接口。
%    \begin{macrocode}
\NewDocumentCommand \ecnuSetup { m }
  { \keys_set:nn { ecnu } {#1} }
%    \end{macrocode}
% \end{macro}
%
% \begin{macro}{\frontmatter}
% 前置部分开始命令。
%    \begin{macrocode}
\NewDocumentCommand \frontmatter { }
  { \cleardoublepage\pagenumbering{Roman} }
%    \end{macrocode}
% \end{macro}
%
% \begin{macro}{\mainmatter}
% 论文主体部分开始命令。
%    \begin{macrocode}
\NewDocumentCommand \mainmatter { }
  { \cleardoublepage\pagenumbering{arabic} }
%    \end{macrocode}
% \end{macro}
%
% \begin{macro}{\backmatter}
% 论文后置部分开始命令。
%    \begin{macrocode}
\NewDocumentCommand \backmatter { }
  { \clearpage{\thispagestyle{empty}\cleardoublepage} }
%    \end{macrocode}
% \end{macro}
%
% 文档类初始设置。
%    \begin{macrocode}
\keys_set:nn { ecnu / style }
  {
    font      = xits,
    fontCJK   = fandol,
    numbering = arabic,
  }
\keys_set:nn { ecnu / info }
  {
    title           = {(中文标题)},
    titleEN         = {(English Title)},
    author          = {(作者姓名)},
    studentID       = {(学号)},
    department      = {(学院)},
    major           = {(专业)},
    supervisor      = {(导师姓名)},
    academicTitle   = {(导师职称)},
    keywords        = {关键词1, 关键词2, 关键词3},
    keywordsEN      = {keyword1, keyword2, keyword3}
  }
%    \end{macrocode}
%
%
%    \begin{macrocode}
%</class>
%    \end{macrocode}
%
% ^^A \iffalse
%
% \subsection{ecnudoc类}
% 
%    \begin{macrocode}
%<*ecnudoc>
%<@@=>
\ExplSyntaxOff
\LoadClass[a4paper, full]{l3doc}
\RequirePackage{xparse}
\RequirePackage[UTF8, heading, scheme=chinese]{ctex}
\RequirePackage{
  caption,
  graphicx,
  listings,
  unicode-math,
  xcolor,
  enumitem,
  metalogo
}
\RequirePackage[
  top     = 2.5cm,
  bottom  = 2.5cm,
  left    = 4.5cm,
  right   = 2cm,
  headsep = 3mm
]{geometry}

\ctexset {
  section = {
    name        = {第,节},
    format+     = \raggedright,
  },
  paragraph = {
    runin         = false,
  }
}

\setlist[enumerate]{
  topsep=0pt,
  parsep=0pt,
  itemsep=0pt
}
\setlist[itemize]{
  topsep=0pt,
  parsep=0pt,
  itemsep=0pt
}

\newlist{optdesc}{description}{3}
\setlist[optdesc]{%
  font          = \mdseries\small\ttfamily,
  align         = right,
  listparindent = \parindent,
  labelsep      = \marginparsep,
  labelindent   = -\marginparsep,
  leftmargin    = 0pt
}
\definecolor{colorRule}{RGB}{164,31,53}
\lstdefinestyle{style@base}{
  basicstyle        = \small\ttfamily,
  lineskip          = 0 pt,
  framerule         = 1 pt,
  framesep          = 3 pt,
  xleftmargin       = 4 pt,
  breaklines        = true,
  showspaces        = false,
  showstringspaces  = false,
  gobble            = 4,
  escapeinside    = {(*}{*)},
  backgroundcolor   = \color{gray!5},
  commentstyle      = \slshape\color{black!60}
}
\lstdefinestyle{style@shell}{
  style     = style@base,
  frame     = l,
  rulecolor = \color{colorRule},
  language  = bash,
}
\lstdefinestyle{style@latex}{
  style     = style@base,
  frame     = l,
  rulecolor = \color{colorRule},
  language  = [LaTeX]TeX,
}
\lstnewenvironment{latexCode}{\lstset{style=style@latex}}{}
\lstnewenvironment{shellCode}{\lstset{style=style@shell}}{}
\def\breakablethinspace{\hskip 0.16667em\relax}
\DeclareDocumentCommand \kvopt { mm }
  {\texttt{#1\breakablethinspace=\breakablethinspace#2}}
\DeclareDocumentEnvironment{arguments}{}
  {\enumerate[label={\texttt{\#\arabic*:~}}, labelsep=0pt, nolistsep]}
  {\endenumerate}

\PassOptionsToPackage { hidelinks, hypertexnames=false } { hyperref }
\RequirePackage { hyperref }
\def\UrlAlphabet{%
      \do\a\do\b\do\c\do\d\do\e\do\f\do\g\do\h\do\i\do\j%
      \do\k\do\l\do\m\do\n\do\o\do\p\do\q\do\r\do\s\do\t%
      \do\u\do\v\do\w\do\x\do\y\do\z\do\A\do\B\do\C\do\D%
      \do\E\do\F\do\G\do\H\do\I\do\J\do\K\do\L\do\M\do\N%
      \do\O\do\P\do\Q\do\R\do\S\do\T\do\U\do\V\do\W\do\X%
      \do\Y\do\Z}
\def\UrlDigits{\do\1\do\2\do\3\do\4\do\5\do\6\do\7\do\8\do\9\do\0}
\g@addto@macro{\UrlBreaks}{\UrlOrds}
\g@addto@macro{\UrlBreaks}{\UrlAlphabet}
\g@addto@macro{\UrlBreaks}{\UrlDigits}
%</ecnudoc>
%    \end{macrocode}
%
%
%
%<*ecnulogo>
%</ecnulogo>
% \fi
% \end{implementation}